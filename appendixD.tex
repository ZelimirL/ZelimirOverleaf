\appendix
\chapter{Fortran code for Calculating Energies for the 2D problem}
\label{AppendixD}

To run the numerical calculations I used gfortan 5.2.0, running on 2.2 GHz, Intel Code 7, Macbook Pro 15" Laptop (mid 2014). The numerial library used was Lapack, version 3.6.0, compiled on the same machine using the gfortran above.

This is Fortran 90 code for calculating even and odd eigenvalues and consequently the energies.

\begin{verbatim}
! To Compile 
! gfortran -O4 -o h2Plus h2plus.f90 hpsort.f90 -L ./ -l lapacke -l lapack -l blas -l install
program main

implicit none

interface 

	subroutine M_Matrix(mMatrix,max,p)
		integer :: max
		double precision :: p
		double precision,dimension(:,:) :: mMatrix
	end subroutine

	subroutine L_Matrix(lMatrix,max,R,p)
		integer :: max
		double precision :: R,p
		double precision,dimension(:,:) :: lMatrix
	end subroutine

	subroutine CalculateEigenvalues(c,matrix,max,wr,wi)
		character*1 :: c
		double precision,dimension(:,:) :: matrix
		integer :: max,np,count
		double precision,dimension(:) :: wr,wi
	end subroutine 

	subroutine hpsort(n,rra)
		integer :: n
		double precision, dimension(:) :: rra
	end subroutine 

end interface

! Declare local variables
integer,parameter :: max = 500   ! matrix dimesions
integer,parameter :: np = 1000  ! number of p points 
integer,parameter :: state = 0   ! 0=ground state, 1=1st excited, etc.. 
integer,parameter :: numRadii = 1
double precision,dimension(numRadii) :: Radii
double precision,dimension(max,max) :: mMatrix,lMatrix
double precision,dimension(np) :: pGrid
double precision :: R  ! Nuclei distance
double precision,dimension(max,np) :: mEig, lEig 
double precision,dimension(max) :: wr,wi
double precision :: pStep, pMin, pMax
integer :: count,i,j, index
double precision :: p,rr
character(len=6) :: fmt
character(len=16) :: rOut

i = 1;
rr = 20.0d0
do while( (rr <= 5.0d0) .and. (i <= numRadii))
	Radii(i) = rr
  rr = rr + 0.2d0
	i = i + 1
end do

do while(i <= numRadii)
  Radii(i) = rr
	rr = rr + 0.5d0
  i = i + 1
end do

do j = 1, numRadii
	R = Radii(j)

	print *,"R = ", R

	pMin = 0
	pMax = 2.d0*R + 1.d0
	pStep = (pMax - pMin) / dble(np)

	count = 1;
	p = pMin
	do while( count <= np )
		! Initialize M  matrix
		call M_Matrix(mMatrix,max,p)

		!Calculate eigenvalues for M matrirx
		call CalculateEigenvalues('m',mMatrix,max,wr,wi)
  	!	 sort
		do i = 1,max
			if( wi(i) /= 0) then
				wr(i) = -1.0d99
			end if
		end do
					
		call hpsort(max,wr)
		do i = 1,max
				mEig(i,count) = wr(max-i+1)
		end do

		! And L Matrix
		call L_Matrix(lMatrix,max,R,p)
		call CalculateEigenvalues('l',lMatrix,max,wr,wi)
		do i = 1,max
			if( wi(i) /= 0) then
				wr(i) = 1.0d99
			end if
		end do
		call hpsort(max,wr)
  	! wr contains eigenvalues from largest to smallest
		do i = 1,max
			lEig(i,count) = wr(i)
		end do
	
		pGrid(count) = p
		p = p + pStep
		count = count + 1
	end do

	! Now write the eigenvalues to file 
	if( R < 10.0d0 ) then
		fmt = "(F3.1)"
	else
		fmt = "(F4.1)"
	end if
	write(rOut, fmt) R
	open(unit=11,file="Eigenvalues-Even-M-R="//trim(rOut)//".dat")
	open(unit=12,file="Eigenvalues-Even-L-R="//trim(rOut)//".dat")
	!	 Todo: Excited states
	index = 1
	do i=1,np
		write(11,*) pGrid(i), mEig(1,i),mEig(2,i), mEig(3,i), mEig(4,i), mEig(5,i)
		write(12,*) pGrid(i), lEig(1,i),lEig(2,i), lEig(3,i), lEig(4,i), lEig(5,i)
	end do
	close(11)
	close(12)
end do

end  ! main

subroutine CalculateEigenvalues(id,matrix,max,wr,wi)
! Calculate and return the eigenvalues 
implicit none

character*1 ::id
double precision,dimension(:,:) :: matrix
integer :: max
double precision,dimension(:) :: wr,wi
!local
double precision,dimension(max,max) :: z 
double precision,dimension(:), allocatable :: work
integer :: info,i
double precision,dimension(1,1) :: dummy
integer :: lda,lwork
double precision :: temp

lda = max
lwork = max * 11

allocate( work(lwork) )

call DGEEV('N','N',max,matrix,lda,wr,wi,dummy,1,dummy,1,work,lwork,info)

deallocate(work)

if( info /= 0 ) then
	print *, "Error in info , info = ", info
	call exit(1)
end if

end subroutine

subroutine M_Matrix(mMatrix,max,p)
! create the M matrix
	implicit none
interface
   function Delta(m,k)
     double precision :: Delta
     integer, intent(in) :: m,k
   end function Delta
end interface

	integer :: max
	double precision :: p, val
  double precision,dimension(:,:) :: mMatrix
	integer :: m,k

	do m = 0,max-1
 		do k = 0,max-1
!     Even state
			val = ( (-1)*((2.0d0 * k)**2) + ( (p**2.0)/2.0d0) ) * Delta(m,k) + & 
!     For the Odd state use
!     val = ( (-1)*((2.0d0 * k + 1)**2) + ( (p**2.0)/4.0d0) ) * Delta(m,k) + &
      ( (p**2)/4.0d0 ) * ( Delta(m,k+1) + Delta(m,k-1) )
			mMatrix(m+1,k+1) = val
		end do
	end do
	
!     Even state
	mMatrix(1,2) = 2 * mMatrix(1,2)
!     For the Odd state use
! mMatrix(1,1) = -1 + (p**2.)/4.0d0
	
end subroutine

subroutine L_Matrix(lMatrix,max,R,p)
! create the L matrix
	implicit none
interface
   function Delta(m,k)
     double precision :: Delta
     integer, intent(in) :: m,k
   end function Delta

  function SubSum(m,k)
      double precision :: SubSum
      integer, intent(in) :: m,k
   end function SubSum
end interface

	integer :: max
	double precision :: p,R,val
	double precision, parameter :: PI = 3.141592653589793238462643d0
  double precision,dimension(:,:) :: lMatrix
	integer :: m,k

	do m = 0,max-1
		do k = 0,max-1
      val = ((-2.d0)*p*k*(2*k+1) - (2*p-1)*(k**2) - 4*p*k + &
			(2*R-p)*(2*k+1)-p -(p**2)+2*R) * Delta(m,k) + &
      (2*p*k*(k+1) - (2*R-p)*(k+1))*Delta(m,k+1) + &
      (2.0d0*p*(k**2) + (2.0d0*p - 1)*k*(2.0d0*k - 1) +(4.0d0*p - 2)*k-&
			(2*R - p)*k)*Delta(m,k-1) - &
      (2.0d0*p - 1)*k*(k - 1)*Delta(m,k-2) + SubSum(m,k)
			lMatrix(m+1,k+1) = (-1)*val
		end do
	end do
	
end subroutine

function SubSum(m,k)
implicit none
interface
   function Delta(m,k)
     double precision :: Delta
     integer, intent(in) :: m,k
   end function Delta
end interface

double precision :: SubSum
integer, intent(in) :: m,k
integer :: i

SubSum = 0.0d0

do i = 0,k-1
   SubSum = SubSum + Delta(m,i)
end do

end function SubSum

function Delta(m,n)
implicit none
double precision :: Delta
integer, intent(in) :: m,n

if (m == n) then
    Delta = 1.d0
else
   	Delta = 0.d0
end if
end function Delta

\end{verbatim}

