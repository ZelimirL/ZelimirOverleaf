\chapter{CT Processes in Two Dimensions}
\label{chap:2DS}
In this part of the thesis, we will consider the following resonant charge transfer process 
$$ H^{+} + H \rightarrow H + H^{+} $$
i.e., the exchange of an electron in the collision of a proton
with a hydrogen atom. This fundamental CT process has been
studied extensively in three dimensions, but in this dissertation, we, for the first time, calculate the rate for
(xx) in a two-dimensional substrate, i.e., both the electron
and nuclei are constrained to move in two dimensions.
The motivation for this study stems from advances in laboratory
studies of 2D systems and the fabrication of quantum materials in 2D. Thus, fundamental collision processes must be investigated theoretically in 2 dimensions. 
2D materials hold great promise for innovative applications. To advance this emerging technology, it is essential to gain a fundamental understanding of the processes that govern the behavior of matter at the atomic and molecular levels. In this thesis, we will examine the interactions of ions with neutral matter in a 2D environment. 
Before we focus on the calculation of Eq. (xx), we review, below, some advances in the physics of 2D systems.
\subsection{Overview of 2D material science}
2D materials can exhibit unique topological properties that do not exist in the 3D case. An exciting and important feature in this context is the emergence of anyons \cite{anyonsR}. Anyons are neither Fermions nor Bosons. The exchange of two anyons induces an arbitrary relative phase between the partners; an effect called fractional statistics.
In 3D space, the relative phase can only take the integer values of $2 \pi $ (Bosons) or $ \pi$ (Fermions).
 \subsection{On 2D Materials No-Go Theorems and Graphene}
Landau and Pierls\cite{LandauG}\cite{Pierls} initially argued that 2D materials could not exist. That argument posited that the displacement of atoms at finite temperatures induces thermal fluctuations on the order of the atomic spacing, thereby melting the 2D material above 0 K. 

There is an even stronger statement against 2D materials, in the form of the Mermin-Wagner-Hohenberg (MWH) theorem \cite{Hohenberg}\cite{Mermin2}. In these papers, the authors show that 2D systems with short-range interactions are unstable and cannot exist. Using thermodynamic arguments, one can show that a 3D system is stable if its free energy is bounded from below and convex (upwards). In this way, the system can minimize the ground-state energy. Because higher-order fluctuations are finite, the system is stable.
The MWH argument posits that, in 2D systems, fluctuations around the ordered state (lattice in the 2D case) decorrelate over large distances, thus destroying the large-scale order. 
Therefore, it seemed that the possible existence and applications of the 2D materials were quite limited. The only 2D materials considered stable were thin molecular films, which could form on a solid's surface. Dash\cite{2DMatter1} predicted that 2D materials exist as a thin film either on the surface of the boundaries of 3D materials.

Despite those arguments, it has been speculated that stable 2D materials are possible\cite{2DMatter1}. The situation changed dramatically in 2004 when 2D carbon (graphene) was discovered and created in the laboratory\cite{Graphene0, GrapheneN} at the University of Manchester, Great Britain. 

2D materials have qualitatively different topological properties\cite{2DMatterCurvature}. Moreover, those properties do not depend on the type of microscopic interactions between particles. While this subject is beyond the scope of this thesis, it does show that the field of 2D materials is a fertile ground for both fundamental and applied research. Interestingly, it has been shown \cite{GraphLayers} that it takes 10 graphene layers for it to begin behaving like a 'regular' 3D material.  
The existence of graphene does not contradict the above-mentioned no-go theorem \cite{Mermin2}, but it does circumvent it \cite{GrapheneRiples}. In graphene, thermal oscillations produce ripples or bending of the graphene sheet. That way, graphene behaves like an elastic membrane that allows phonons to propagate in 2 dimensions and couple in a plane stretching along the transverse fluctuations out of the plane. These phonons mediate a long-range interaction, thereby circumventing the Mermin-Wagner no-go theorem. 

Additionally \cite{Graphene0} \cite{Dirac2}\cite{Dirac3},2D materials exhibit several other interesting properties. Its quantum excitations behave in accordance with relativistic dynamics. In condensed matter systems, the solutions of the Schrödinger equation describe the material's electronic properties. In graphene, the electrons obey the non-relativistic Schrodinger equation, but interactions with the graphene lattice's periodic potential give rise to solutions referred to as quasiparticle excitations, and they behave in the same manner as massless Dirac fermions. Even at low energies, these quasiparticles are described by a Dirac equation in 2+1 dimensions.  

\subsection{Excitons}
These excitations represent a wide class of intrinsic electronic excitations (Excitons) \cite{Excitons1, Excitons2} in crystals of semiconductors and dielectrics. In one model, excitons are bound states of an electron and a hole, typically formed when an incident photon is absorbed, thereby exciting an electron from the valence to the conduction band. The attractive Coulomb interaction between the excited electron and the newly created hole binds them together to form a bound neutral compound system of the two charge carriers, similar to a hydrogen atom. 
The character of exciton motion depends on the strength of the exciton interaction with phonons. The electron-hole model is common in insulating and semiconducting crystals.
Ya. I. Frenkel (1931) introduced excitons to explain the light absorption in crystals, which does not lead to photoconductivity. In the Frenkel model, the exciton is considered as an electronic excitation of one crystal site with the energy close to, but a bit smaller than, that necessary for the excitation of a free electron. Due to the crystal's translational symmetry, the exciton can move along the lattice sites, transferring energy to electrically active or luminescent centers.

Similar excitations occur in 2D systems. \cite{Excitons2D1,Excitons2D3}. 2D excitons promise to serve as building blocks for 2D electronics. \cite{Excitons2D2}.  Although exciton-based transistor action has been demonstrated in bulk semiconductor coupled quantum wells, the low operating temperature required limits their practical application. The emergence of 2D semiconductors with large exciton binding energies may lead to excitonic devices and circuits that operate at room temperature. \cite{Excitons2D2}.

\subsection{Graphene}
Graphene exhibits similar properties\cite{Graphene0}, such as its Quantum Electrodynamics (QED) like electronic spectrum, electron tunneling. Another interesting consequence of the QED-like electron spectrum is the possibility of experimentally studying QED in curved space by controllable bending of a graphene sheet. Such a study may offer a possibility to address a certain class of cosmological problems. 

At present, since the discovery of graphene, several new 2D materials have been identified\cite{Many2DMaterials}. 
As new 2D materials are discovered, researchers are investigating their potential applications. One of the most useful applications of exotic materials is electronics. These new 2D materials promise that they can be used to build existing electronic components, but with potentially better performances \cite{2DEJour1}\cite{2DEJour2}.

Nanotubes are rolled-up graphene sheets, and depending on the application, graphene (as a sheet) is sometimes superior, sometimes inferior, and sometimes completely different. Another notable application is the emergence of high-temperature superconductivity in 2D materials\cite{2DSuper}.

\subsection*{2D Electronics}
Driven by a demand for higher performance and lower consumption in electronic devices, 2D materials have found application in that arena\cite{2DEJour1},\cite{2DEJour2}. 
There is the hope that graphene can be used to build active electronic components, such as diodes, transistors (FET), and other electronic components. Alternatively, graphene can be used as a conductor in electronic components and batteries owing to its high electrical conductivity. Graphene itself is a zero-gap semiconductor. There is another unique opportunity in 2D semiconductors. In a 3D semiconductor, electronic states are buried deep inside the material. Unlike the surface state, electron states within the material are localized, i.e., their wave functions have exponentially decaying tails.

In 2D semiconductors, the electronic states exist
on the surface. The surface states admit both localized and non-localized block-bulk states \cite{LocStates}.
Another interesting application of 2D electronics is the LED elements in 2D \cite{2DLED}. Electronics research is closely related to research on new 2D materials. For example, graphene itself is not compatible with silicon. However, there is a new material, black phosphorus\cite{2DPhos}, which is both compatible with silicon and it holds promise for future electronic\cite{2DPhos3} devices. This fact relies on two key properties: the first is that black phosphorus has higher mobility, the speed at which it can carry an electrical charge, than silicon. The other is that it has a finite bandgap, whereas graphene does not. So, in essence, in the presence of an electric field, black phosphorus can act as a semiconductor\cite{2DPhos2}.

\subsection*{Quantum Computing}
One promising platform, topological quantum computing \cite{Tqc1,Tqc2,Tqc3, Tqc4, Tqc5}, relies on the unusual quantum properties of 2-dimensional (2D) matter.
The current computer architecture is based on the so-called 'von Neumann' architecture. The algorithms based on that architecture have inherent limitations, and, as it currently stands, the architecture cannot overcome them.

There are two main classes of computer algorithms, P and NP. There are subclasses within those classes, but those two classes capture the essential behavior of classical algorithms. The algorithms belonging to class P are solvable in polynomial time, with regard to the size of the input. Those algorithms belonging to class NP are those that require exponential time to complete\cite{PvsNP}.

Currently, the major unsolved problem in Computer Science is the P vs. NP problem \cite{PvsNP}. While the hypothesis that $ P \neq NP $ is generally accepted, no proof has been found. Applying quantum-mechanical principles in engineering disciplines, such as computer science \cite{FQC}, promises to yield a qualitatively more powerful quantum computer. 

 Now, all problems in the NP class (more precisely, NP-complete subclass) can be transformed into each other in polynomial time \cite{NPComplete},  therefore finding a polynomial-time solution for one of them would amount to having a polynomial-time solution to all problems in the NP class. So the goal is rather worthy. 

It is currently known that quantum computers cannot solve NP problems. They could solve the class of the bounded-error, quantum, polynomial time (BQP) problems \cite{BQP}. 
Even if the NP problems remain intractable, just the possibility that, according to Feynman \cite{FQC}, a quantum computer can be used to simulate the quantum processes, which will in itself be a step forward. 

For example, Schor's algorithm \cite{Schor} can provably solve integer factorization in polynomial time, as opposed to the best classical algorithm (quadratic sieve), which works in sub-exponential time \cite{Pomerance}. Unfortunately, the integer factorization problem, as well as other BQP problems, do not belong to the class of NP-complete problems. 

Interestingly, according to Abrams and Lloyd \cite{Abrams_1998}, \cite{Abrams_1999}, if the small nonlinear term is added to Quantum Mechanics, quantum computers would be able to solve NP-complete problems in polynomial time.

\subsection*{Topological Quantum Computing}
The concept of topological quantum computing \cite{Tqc1}
depends on the unique properties of 2D materials. In principle, there are no theoretical objections to building a quantum computer; there are serious technical issues, such as noise and scaling limitations, for current qubit technologies that need to be overcome\cite{QCProblems}. 
The topological approach promises a physically realizable quantum computer\cite{Tqc2,Tqc3}. 
\subsection{Overview of CT theory for low energy collisions in 3D}

\subsection*{Anyons}
Anyons are quasi-particles that can arise in 2D systems and whose quantum statistics in neither Fermionic nor Bosonic \cite{Anyons1}. This fact has been proven\cite{Anyons2}, but
arbitrary statistics are valid only in 2 dimensions \cite{Walsh}. After exchanging two identical particles, the wave function gains an arbitrary phase factor $ \Psi = e^{i\theta}\Psi $, where $ \theta = 2\pi\nu^{s} $. In contrast, for a 3D system, the exchange of two particles leads to either phase factor $ \pi $ or $ 0 $, for fermions and bosons, respectively.

In general, there are two kinds of Anyons: Abelian and non-Abelian. Abelian Anyons have been reliably detected in systems exhibiting the Fractional Quantum Hall Effect\cite{FQHE}. Whereas the computation with Abelian Anyons is theoretically possible \cite{AbelianAnyons}, considerably more attention has been paid to non-Abelian Anyons.
There are several possibilities for realizing non-Abelian statistics experimentally: 
\begin{itemize}
  \item A two-dimensional electron gas in a large magnetic field (the fractional quantum Hall effect)
 \item Rapidly rotating Bose-Einstein condensates\cite{RrBeC} 
 \item Frustrated magnets \cite{FrMag}.
 \item One of the anyons candidates the Majorana fermion \cite{Majorana}. 
\end{itemize}
Majorana fermion (particle) is a particle that is its own antiparticle. A pair of localized Majorana states has been predicted to reside at the ends of a specially designed superconducting wire. Very recently (2020) \cite{Manna8775}
there is a possibility that Majorana particles has been observed for the first time. The other property of the Majorana fermion is that it is a non-Abelian anyon.
This presents the possibility for the realization of fault-tolerant topological quantum computing\cite{AnyonsTqc}. 



\subsection*{Excitons and Trions}
Other interesting quasiparticle phenomena are called excitons \cite{Excitons2D1}. It is a bound state of a hole and an electron in a solid-state material. Excitons enable energy transport without charge transport. There are two types of excitons: 

1. Frenkel excitons are found in organic molecular crystals\cite{Excitons3} and have binding energies on
order of $ 1eV $ and radii $ ~ 10\text{\AA} $.  


2. Wannier excitons are found in semiconductors\cite{Excitons2} and have binding energies on the order of $ ~1 meV $ with radii$ ~100\text{\AA}. $


Excitons have been observed in 2D materials\cite{Excitons2D2} as
well as 2D excitons in 3D materials. The promise is that these quasi-particles can enable the development of new electronic components, particularly optoelectronic ones.

Since the exciton system resembles either the hydrogen atom or the hydrogen molecule, it seems conceivable that the dynamics of the $ H_2^{+} $ molecule is qualitatively similar and could be applied to the exciton systems as well.

A similar system is the trion, a bound state of three charged particles. It consists of either 2 electrons and a hole, or 2 holes and an electron, therefore resembling the hydrogen molecular ion. 



\section{Molecular Orbitals in 2 D}
In order to calculate the resonant charge transfer rate (RCT), we require the potential surfaces for the two lowest ground state Born-Oppenheimer (BO) that correlate to the degenerate asymptotic
limits of the proton-hydrogen system. Many open-source and proprietary tools are available to calculate the latter in three dimensions; to the best of our knowledge, none of such tools are available for the 2D case. In this thesis, we use a first-principles approach to calculate accurate BO potential surfaces for application in our collision studies.

Prior to that discussion, we present a qualitative description and derive approximate solutions using the LCAO method to examine the gerade and ungerade potentials for the ground state of the $H_{2}^{+}$ ion.
\subsection{Gerade and Ungerade Molecular Potentials}
In diatomic molecules, gerade (German for even) and ungerade (German for odd) describe how the electronic wavefunction behaves under inversion symmetry, i.e., symmetry of the molecular orbital with respect to the inversion center.
If the internuclear axis is placed through the center of the molecule, the inversion manifests with every coordinate point $ r $ with $ -r $.
These terms are meaningful only for molecules or systems that possess a center of inversion, points such that $ r\,\rightarrow -r $ maps the molecule onto itself.
The definition:
\begin{itemize}
  \item Gerade(g): an orbital wavefunction $ \Psi(r) $ is gerade if it is unchanged by inversion: $ \Psi(-r) = \Psi(r) $. It has even parity. Example: $ 1s\sigma_g $

  \item Ungerade(u): an orbital wavefunction $ \Psi(r) $ is ungerade if it changes sign under inversion: $ \Psi(-r) = -\Psi(r)  $. It has odd parity. Example $ 2p\sigma_u $.
\end{itemize}

\section{Selection rules and consequences}
Electric dipole transitions: transitions between electronic states follow the parity selection rule. The electric dipole operator is odd (Ungerade). Therefore, allowed transitions must change parity: $ g \rightarrow u $. Transitions $ g \rightarrow g $ or $ u \rightarrow u $ are electric dipole forbidden (though allowed by higher multipoles or vibronic coupling).
\\
Orbital interactions: orbitals can only mix (form nonzero overlap/hybridize) if they belong to the same symmetry species, including parity. Thus, g orbitals do not mix with u orbitals.


\section{LCAO}
The ground state expression gives amplitude for the 2-dimensional hydrogen atom \cite{YangXL}
\begin{equation}
\psi_{1s}({\bf r}) =\frac{4}{\sqrt{2 \pi}}e^{-2 \, r}
\end{equation}
where $ r = |{\bf r}|$. In the LCAO method, an approximate description of the 2D gerade molecule $H_{2} $, for internuclear separation ${\bf R}$ is given by
\begin{equation}
\psi_{g}({\bf r},{\bf R}) = C(R) \Bigl ( \psi_{1s}({\bf r}-{\bf R}/2 ) + \psi_{1s}({\bf r}+{\bf R}/2 ) \Bigr )
\end{equation}
where $C(R)$ is a normalization factor so that
\begin{equation}
\int d^2 {\bf r} \, |\psi_{g}({\bf r},{\bf R})|^2 = C^2(R) \Bigl (
2 + 2 \int d^2 {\bf r} \, \psi_{1s}({\bf r}) \psi_{1s}({\bf r}+{\bf R}) \Bigr ) =1
\end{equation}
or,
\begin{equation}
  C(R) = \frac{1}{\sqrt{2 (1+ S(R) )}} \text{   and} \quad S(R) \equiv \int d^2 {\bf r} \, \psi_{1s}({\bf r}) \psi_{1s}({\bf r}+{\bf R}) \label{eq:soverlap}
\end{equation}
We now evaluate the expectation values $ \langle V \rangle $ of
the potential
\begin{equation}
-\frac{1}{|{\bf r} - {\bf R}/2|} - \frac{1}{|{\bf r} + {\bf R}/2|}.
\end{equation}
Inserting the above wave amplitude to evaluate the expectation value, we get
\begin{equation}
\langle V \rangle_{g} =
-2 \int \frac{d^2 {\bf r}}{r} \, |\psi_{g}({\bf r}+{\bf R}/2 )|^2.
\end{equation}
We use the virial theorem to estimate the gerade ground state expectation value for the kinetic energy operator
\begin{equation}
\langle KE \rangle_{g} = \frac{1}{2} \langle {\bf r} \cdot {\bf \nabla} V \rangle
\end{equation}
and since

\begin{equation}
{\bf r} \cdot {\bf \nabla} V  = {\bf r}\cdot \frac{( {\bf r}- {\bf R}/2) }{ |{\bf r}- {\bf R}/2 |^{3} } + {\bf r}\cdot \frac{( {\bf r}+ {\bf R}/2) }{ |{\bf r}+ {\bf R}/2 |^{3} }
\end{equation}
we find
\begin{equation}
\langle KE \rangle_{g} =
 \int \frac{d^2 {\bf r}}{r} \, |\psi_{g}({\bf r}+{\bf R}/2 )|^2
+ \int \frac{d^2 {\bf r}}{r^3} \, ({\bf r}\cdot{\bf R}/2) \,
|\psi_{g}({\bf r}+{\bf R}/2 )|^2 .
\end{equation}
In the same manner, we obtain the BO energy for the ungerade BO state,
\begin{equation}
\psi_{u}({\bf r},{\bf R}) =\frac{1}{\sqrt{2 (1- S(R) )}}\int d^2 {\bf r} \, \Bigl ( \psi_{1s}({\bf r}-{\bf R}/2 ) - \psi_{1s}({\bf r}+{\bf R}/2 ) \Bigr ).
\end{equation}

In Table \ref{tab:lcaotabG}, we present the accurate and LCAO values for the gerade state.

Graphical comparison of the accurate vs LCAO potential for the gerade state. The exact potential is shown in Red, and the approximation is shown in blue.

\includegraphics{"Vg-Vs-LCAO-good"}

This is more readily observed in the enhanced plot.

\includegraphics{"Vg-Vs-LCAO-2"}

The above graphs show the accurate gerade potentials (red dots) versus those obtained with the simple LCAO method described above.
We observe that for $R > 10$, there is fair agreement; the agreement is poorer for $R < 10$. This is a reasonable result, as the LCAO, which includes only the ground-state orbital, is a poor description in the molecular region.

We now show the same comparison for the ungerade state in Table \ref{tab:lcaotabU} we show the accurate vs LCAO values for the ungerade state.

Graphical comparison of the accurate and LCAO potentials for the ungerade state.

\includegraphics{"Vu-Vs-LCAO-good"}

Unlike for the gerade state, in the ungerade case the LCAO is quite a good approximation to the accurate potential. \\
The reason is the symmetries of the gerade and ungerade states and the overlap integral \eqref{eq:soverlap}.
In the ungerade case we have:\\
$ \psi_{u}({\bf r},{\bf R}) \approx  \psi_{1s}({\bf r}-{\bf R}/2 ) - \psi_{1s}({\bf r}+{\bf R}/2 ) $ which makes $ 1 - S \approx 1 $.\\
It follows that the orbital are close to being orthogonal, which makes LCAO approximation more accurate.\\

In the gerade case, we have $ \psi_{g}({\bf r},{\bf R}) \approx  \psi_{1s}({\bf r}+{\bf R}/2 ) - \psi_{1s}({\bf r}+{\bf R}/2 ) $. In this case, we have $ 1 + S > S $. In this case, the LCAO orbits are much less orthogonal, so the overlap error becomes much greater.

In the following figures, we show the atomic orbitals for the gerade and ungerade cases as a function of the internuclear distance R.


%\includepdf[pages=-,pagecommand=\subsubsection{Gerade potentials as function of the internuclear distance R},offset=0 -1.5cm]{Ger-Lcao-Graphs.pdf}
Gerade:

\includegraphics[width=1\textwidth]{Ger-Lcao-Graphs}

Gerade 3D:

\includegraphics[width=1\textwidth]{Ger-Lcao-3D-Graphs}

Ungerade:

\includegraphics[width=1\textwidth]{UnGer-Lcao-Graphs}

Ungerade 3D:

\includegraphics[width=1\textwidth]{UnGer-Lcao-3D-Graphs}

Following our previous discussion, we have determined that our LCAO example provides only a rough qualitative description of the two-dimensional ground state potential surfaces for the \(H_{2}^{+}\) molecule. To obtain more accurate data, we will introduce a semi-analytic procedure initially developed by Bates and his colleagues for the treatment of the three-dimensional \(H_{2}^{+}\) molecular ion in the upcoming chapters. 

In the next chapter, we will review this method as it applies to the three-dimensional system and provide a more precise description of the data presented in Ref. \cite{bates}. Subsequently, we will use a similar approach to generate the data for the two-dimensional case of \(H_{2}^{+}\). 

We will compare our results with those from a previous study (cite) and apply our findings in the calculation of the RCT process given by Eq. (xx) at ultra-cold temperatures.





%\includepdf[pages=-,pagecommand=\subsubsection{Ungerade potentials as function of the internuclear distance R}, offset=0 -1.5cm]{UnGer-Lcao-Graphs.pdf}


% \section{Resonant Charge Transfer}

% In the Resonant Charge Transfer, the electron tunnels through the molecular barrier between two protons. There is no emission of the photon. Since no energy is lost, $ E_{initial} = E_{final} $ the process is elastic.

% The process is $ H^+ + H \rightarrow H + H^+ $.

% Because the two protons are identical, the molecular ion, $ H_2^+ $ in our case,  has two adiabatic electronic states:
% \begin{itemize}
%   \item{gerade $ g $}
%   \item{ungerade $ u $ }
% \end{itemize}

% During the collision, the nuclear motion accumulates different phase shifts on the gerade and ungerade potentials.
% When the nuclei separate again, the superposition is no longer symmetric, so the electron localizes on either nucleus with some probability.\\
% This produces charge exchange without radiation, and the process is controlled by the phase shift difference between the gerade and ungerade states.

% Here we shall investigate the low energy collision process between the hydrogen atom and the hydrogen ion, in 2 dimension. We use fully quantum mechanical approach to calculate the cross section and the emission spectra of the reaction $ H(2\prescript{2}{}S) + H(1\prescript{1}{}S)\rightarrow H(1\prescript{1}{}S) + H(1\prescript{1}{}S) + \hbar\omega $, formed by the quenching of the excited $ H $ atom by the approaching $ H $ atom, in 2D dimensions. To my knowledge, there is no experimental results related to the 2D problem. There are results for the 3D case, listed in \cite{Zygelman88} and references there.

% In this calculation, the atoms are confined in 2 dimensions, while the radiation is emitted or absorbed in all 3D space. Therefore the photon can be emitted in any direction and the potential felt by the incoming ion is the standard 3D Coulomb potential.

% We will model this as a scattering problem, using a Born-Oppenheimer approximation and optical theorem. The Hamiltonian in this case will contain another term, namely the interaction of the radiation field with the electron

% While the problem has not been solved analytically, we will show that it can be solved numerically, to a desired precision. The radiative processes are driven by the interaction of the collision system with the radiation field. The direct change transfer is due to the transition between atomic (molecular) states due to the nuclear motion. Because the collision energy considered is low, typically only molecular states included are those which correspond to the initial $ A \prescript{1}{}\Sigma^+ $  and final $ X \prescript{1}{}\Sigma^+ $ channels.

% We shall investigate the low temperature collision process between the hydrogen atom and the hydrogen ion, in 2 dimension. We use fully quantum mechanical approach to calculate the cross section and the emission spectra of the reaction $ H(2\prescript{2}{}S) + H(1\prescript{1}{}S)\rightarrow H(1\prescript{1}{}S) + H(1\prescript{1}{}S) + \hbar\omega $, formed by the quenching of the excited $ H $ atom by the approaching $ H $ atom, in 2D dimensions. To my knowledge, there is no experimental results related to the 2D problem. There are results for the 3D case, listed in \cite{Zygelman88} and references there.

% In this calculation, the atoms are confined in 2 dimensions, while the radiation is emitted or absorbed in all 3D space. Therefore the photon can be emitted in any direction and the potential felt by the incoming ion is the standard 3D Coulomb potential.

% We will model this as a scattering problem, using a Born-Oppenheimer approximation and optical theorem. The Hamiltonian in this case will contain another term, namely the interaction of the radiation field with the electron

% From now on  we shall start with the calculation. We shall replicate the existing results, but try a new method.
