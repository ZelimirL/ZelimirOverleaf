\chapter{2D Systems}
\label{chap:2DS}
Here we focus on the radiative processes in 2 dimensions. The discussion from the previous chapter applies here.
To repeat we consider the atoms and molecules confined in 2 dimensions while the radiation is in 3 dimensions. as 'usual'.

\section{Description}

....

\section{Approximation}

Here we briefly derive the approximate solution, in order to get a 'feel' of what we are looking into.  We want the show the molecular potential of the $ H_2^+ $ ion.
We use the LCAO method and examine the Gerade and Ungerade potentials.

\subsection{Gerade and Ungerade Molecular Potentials}
In diatomic molecules terms Gerade (German for even) and Ungerade (German for odd)describe how the electronic wavefunction behaves under inversion symmetry, i.e. symmetry of the molecular orbital with respect to the inversion center.
If the place the internuclear axis passing through the center of the molecule, the inversion manifests with every coordinate point $ r $ with $ -r $ .
These terms re meaningful only for molecules or systems that possess a center of inversion, points such that $ r\,\rightarrow -r $ maps the molecule onto itself.
The definition:
\begin{itemize}
  \item Gerade(g): an orbital wavefunction $ \Psi(r) $ is gerade if it is unchanged by inversion: $ \Psi(-r) = \Psi(r) $. It has even parity. Example: $ 1s\sigma_g $

  \item Ungerade(u): an orbital wavefunction $ \Psi(r) $ is ungerade if it changes sign under inversion: $ \Psi(-r) = -\Psi(r)  $. It has odd parity. Example $ 2p\sigma_u $.
\end{itemize}

\subsection{Selection rules and consequences}
Electric dipole transitions: transitions between electronic states follow parity selection rule. Electric dipole operator is odd (Ungerade). Therefore allowed transitions must change parity: $ g \rightarrow u $. Transitions $ g \rightarrow g $ or $ u \rightarrow u $ are electric dipole forbidden (though allowed by higher multipoles or vibronic coupling).
\\
Orbital interactions: orbitals can only mix (form nonzero overlap/hybridize) if they belong to the same symmetry species, including parity. Thus g orbitals do not mix with u orbitals.


\subsection{LCAO}
The ground state expression gives amplitude for the 2-dimensional hydrogen atom \cite{YangXL}
\begin{equation}
\psi_{1s}({\bf r}) =\frac{4}{\sqrt{2 \pi}}e^{-2 \, r}
\end{equation}
where $ r = |{\bf r}|$. In the LCAO method, an approximate description of the 2D gerade molecule $H_{2} $, for internuclear separation ${\bf R}$ is given by
\begin{equation}
\psi_{g}({\bf r},{\bf R}) = C(R) \Bigl ( \psi_{1s}({\bf r}-{\bf R}/2 ) + \psi_{1s}({\bf r}+{\bf R}/2 ) \Bigr )
\end{equation}
where $C(R)$ is a normalization factor so that
\begin{equation}
\int d^2 {\bf r} \, |\psi_{g}({\bf r},{\bf R})|^2 = C^2(R) \Bigl (
2 + 2 \int d^2 {\bf r} \, \psi_{1s}({\bf r}) \psi_{1s}({\bf r}+{\bf R}) \Bigr ) =1
\end{equation}
or,
\begin{equation}
  C(R) = \frac{1}{\sqrt{2 (1+ S(R) )}} \text{   and} \quad S(R) \equiv \int d^2 {\bf r} \, \psi_{1s}({\bf r}) \psi_{1s}({\bf r}+{\bf R}) \label{eq:soverlap}
\end{equation}
We now evaluate the expectation values $ \langle V \rangle $ of
the potential
\begin{equation}
-\frac{1}{|{\bf r} - {\bf R}/2|} - \frac{1}{|{\bf r} + {\bf R}/2|}.
\end{equation}
Inserting the above wave amplitude to evaluate the expectation value, we get
\begin{equation}
\langle V \rangle_{g} =
-2 \int \frac{d^2 {\bf r}}{r} \, |\psi_{g}({\bf r}+{\bf R}/2 )|^2.
\end{equation}
We use the virial theorem to estimate the gerade ground state expectation value for the kinetic energy operator
\begin{equation}
\langle KE \rangle_{g} = \frac{1}{2} \langle {\bf r} \cdot {\bf \nabla} V \rangle
\end{equation}
and since

\begin{equation}
{\bf r} \cdot {\bf \nabla} V  = {\bf r}\cdot \frac{( {\bf r}- {\bf R}/2) }{ |{\bf r}- {\bf R}/2 |^{3} } + {\bf r}\cdot \frac{( {\bf r}+ {\bf R}/2) }{ |{\bf r}+ {\bf R}/2 |^{3} }
\end{equation}
we find
\begin{equation}
\langle KE \rangle_{g} =
 \int \frac{d^2 {\bf r}}{r} \, |\psi_{g}({\bf r}+{\bf R}/2 )|^2
+ \int \frac{d^2 {\bf r}}{r^3} \, ({\bf r}\cdot{\bf R}/2) \,
|\psi_{g}({\bf r}+{\bf R}/2 )|^2 .
\end{equation}
In the same manner, we obtain the BO energy for the ungerade BO state,
\begin{equation}
\psi_{u}({\bf r},{\bf R}) =\frac{1}{\sqrt{2 (1- S(R) )}}\int d^2 {\bf r} \, \Bigl ( \psi_{1s}({\bf r}-{\bf R}/2 ) - \psi_{1s}({\bf r}+{\bf R}/2 ) \Bigr ).
\end{equation}

In Table \ref{tab:lcaotabG} we show the accurate vs LCAO values for the gerade state.

Graphical comparison of the accurate vs LCAO potential for the gerade state. The accurate potential is in Red and the approximation is in blue.

\includegraphics{"Vg-Vs-LCAO-good.png"}

This can be easier observed in the enhanced plot.

\includegraphics{"Vg-Vs-LCAO-2.png"}

Above graphs shows accurate gerade potentials (red dots), versus the ones obtained with the simplistic LCAO method described above.
We can observe that for $ R > 10 $ there is fair agreement; not so good for $ R < 10 $. This is a reasonable result as the LCAO including only the ground state orbital is a poor description in the molecular region.

We now show the same comparison for the ungerade state in Table \ref{tab:lcaotabU} we show the accurate vs LCAO values for the ungerade state.

raphical comparison of the accurate vs LCAO potential for the ungerade state.

\includegraphics{"Vu-Vs-LCAO-good.png"}

Unlike for the gerade state, in the ungerade case the LCAO is quite a good approximation to the accurate potential. \\
The reason is the symmetries of the gerade and ungerade states and the overlap integral \eqref{eq:soverlap}.
In the ungerade case we have:\\
$ \psi_{u}({\bf r},{\bf R}) \approx  \psi_{1s}({\bf r}-{\bf R}/2 ) - \psi_{1s}({\bf r}+{\bf R}/2 ) $ which makes $ 1 - S \approx 1 $.\\
It follows that the orbital are close to being orthogonal, which makes LCAO approximation more accurate.\\

In the gerade case, we have $ \psi_{g}({\bf r},{\bf R}) \approx  \psi_{1s}({\bf r}+{\bf R}/2 ) - \psi_{1s}({\bf r}+{\bf R}/2 ) $. In this case we have $ 1 + S > S $. In this case the LCAO orbits are much less orthogonal, so the overlap error becomes much greater.

In the following figures we show the atomic orbitals for the gerade and ungerade case as a function of the internuclear distance R.


%\includepdf[pages=-,pagecommand=\subsubsection{Gerade potentials as function of the internuclear distance R},offset=0 -1.5cm]{Ger-Lcao-Graphs.pdf}
Gerade:

\includegraphics[width=1\textwidth]{Ger-Lcao-Graphs.jpg}

Gerade 3D:

\includegraphics[width=1\textwidth]{Ger-Lcao-3D-Graphs.jpg}

Ungerade:

\includegraphics[width=1\textwidth]{UnGer-Lcao-Graphs.jpg}

Ungerade 3D:

\includegraphics[width=1\textwidth]{UnGer-Lcao-3D-Graphs.jpg}

%\includepdf[pages=-,pagecommand=\subsubsection{Ungerade potentials as function of the internuclear distance R}, offset=0 -1.5cm]{UnGer-Lcao-Graphs.pdf}


