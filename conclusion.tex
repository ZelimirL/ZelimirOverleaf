\chapter{Conclusion}
\label{conclusion}

\section{Comparison with 3D case}

The paper by Dalgarno and  M R C McDowell \cite{DalgarnoMcDowel} reports that the charge transfer cross sections decrease from $ 2\times 10^{-14} cm^2 $ at an impact energy of 1 eV to $ 0.3\times 10^{-14} cm^2 $ at 1000 eV and the diffusion cross sections lead to values of the mobility of H- ions in atomic hydrogen decreasing from $ 3.5\,cm^2\,volt^{-1}\,sec^{-1} $ at 100°K to i$ 1.8\,cm^2\,volt^{-1}\,sec^{-1} $ at 600°K.

Also the paper by B. Zygelman et all \cite{ZL} in the ultra-cold limit shows values of $ A(R)$ in the order of $ 10^6\,s^{-1} $.

In our case, we did calculation for the ultracold limit, from $ 10meV$ to $ 1eV $. The values for the cross section computed are 
show figure~\ref{fig:CTCS}

In this thesis, we considered the 2 dimensional system embedded in the 3 dimensional world. We investigated the scattering processes for particles in 2 dimensions. In Introduction we researched and discussed the application of the 2 dimensional materials, current and some possible future ones. We also discussed the applications of the 2D processes for Quantum Computing.

In 2D, the difference in potential between the gerade and ungerade states $ X^2\Sigma^+ $, $ A^2\Sigma^+ $ respectively is smaller than in the 3D case, so the probability of a transition is greater.

Following the Introduction we focused on the hydrogen molecular ion $ H_2^{+} $. We recalculated the existing result in 3 dimension. 
Then we applied the same technique to calculate the energy levels of the hydrogen molecular ion,$ H_2^{+} $ in 2 dimensions. Our approach produced results quickly but the accuracy was lacking. We used an existing approach to increase the accuracy of the calculation. In chapters 3, 4 we used the results from the chapter 2 to calculate the scattering length, then compute dipole moment and Einstein's A coefficients as function of the inter-nuclear distance for the Radiative Quenching.

The results obtain are different from the 3D case. The potential curves for the $ A^2\Sigma^+ $ and $X^2\Sigma^+ $ are similar but the potential difference between states is smaller. 

The computed value of the Einstein's coefficients is different than in the 3D case, indicating that excited state of the $ H_2^{+} $ is quite unstable. In addition the potential well of the lowest $ X^2\Sigma^+ $ state is quite shallow, so in 2D case even the  $ H_2^{+} $ is less stable than its 3D counterpart.  

The references contain books and articles used to support the above investigation and calculation.

\subsubsection{P.S.}
Copying from one place is plagiarism, copying from multiple places is research.


