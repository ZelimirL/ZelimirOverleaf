
\label{conclusion}

In this thesis, we considered the 2 dimensional system embedded in the 3 dimensional world. We investigated the scattering processes for particles in 2 dimensions. In Introduction we researched and discussed the application of the 2 dimensional materials, current and some possible future ones. We also discussed the applications of the 2D processes for Quantum Computing.

Following the Introduction we focused on the hydrogen molecular ion $ H_2^{+} $. We recalculated the existing result in 3 dimension. 
Then we applied the same technique to calculate the energy levels of the hydrogen molecular ion,$ H_2^{+} $ in 2 dimensions. Our approach produced results quickly but the accurary was lacking. We used an existing approach to increare the accuracy of the calculation. In chapters 3, 4 we used the results from the chapter 2 to calculate the scattering length, then compute diple moment and Einstein's A coefficients as function of the internuclear distance for the Radiative Quenching.

The results obtain are different from the 3D case. The potential curves for the $ A^2\Sigma^+ $ and $X^2\Sigma^+ $ are similar but the potential difference between states is smaller. 

The computed value of the Einstein's coefficients is quite high, indicating that excited state of the $ H_2^{+} $ is quite unstable. In addition the potential well of the lowest $ X^2\Sigma^+ $ state is quite shallow, so in 2D case even the  $ H_2^{+} $ is less stable than its 3D counterpart.  

The references contain books and articles used to support the above investigation and calculation.

P.S.
Copying from one place is plagiarism, copying from multiple places is research


