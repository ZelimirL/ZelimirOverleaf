\chapter{Radiative Processes}
\section{Decription}

The analysis of the radiative processes is related to the analysis of the general scattering process. So we have briefly touched scattering in the previous section.l

Typically for the scattering processes at this level, we can distinguish between the radiative and non-radiative processes. 
Following \cite{ZygelmanCT} we consider a positively charged Hydrogen ion approaching a neutral Hydrogen atom. As the values or R (internuclear distance) becomes smaller, there increases a probability that the electron will escape one atom and attach itself to another.\\
When this happens a charge transfer has occured. \\
There are two classes of transfer, endoergic and exothermic .\\
In the endoergic case the binding energy of the electron on the ion is less than that of the atom. In this case,in order to satisfy the energy conservation requirements, an incoming ion needs to supply the additional energy. However if the ion approaches slowly, there is no extra supply of energy. So when an additional energy is not available, charge transfer can only occur if the final electron binding energy is greater than or equal to its binding energy on the atom.\\
If the energy is equal, the reaction is called resonant charge transfer.\\
if not the reaction will processed with the possibility of the emission of a photon.
If radiation is emitted, we call the reaction radiative charge transfer/association and, if not, direct charge transfer.

Radiative processes involve emission of a photon and they typically mean Radiative Charge Transfer: \\
$ A(n+1\prescript{1}{}S) + B \longleftrightarrow A(n\prescript{1}{}S) + B + \hbar\omega $ \\
and Radiative Association: \\
$ A(n+1\prescript{1}{}S) + B \longleftrightarrow A(n\prescript{1}{}S)B + \hbar\omega $ .\\
For example, the simplest case of the radiative process is the collision of the two hydrogen atoms, in the process $ H + H^+ \rightarrow H^+ + H + \hbar\omega $.\\

While the non-radiative process, with no photon emission, is a Charge Transfer: \\
$ A(n+1\prescript{1}{}S) + B(n\prescript{1}{}S) \longleftrightarrow  A(n\prescript{1}{}S) + B(n+1\prescript{1}{}S) $ .\\

Radiative Quenching is an interaction between two atoms (or molecules), one being in excited state and the other in 'normal' ground state. During this process the excited atom emits a photon and drops to a ground state.
The simplest such case is the collision of the two hydrogen molecules, in the process $ H(2\,{}^2\!S) + H(1\,{}^2\!S) \rightarrow H(1\,{}^2\!S) + H(1\,{}^2\!S) + \hbar\omega $.

There are several theoretical approaches to this problem. \cite{RadQuench1}, \cite{RadQuench2}, \cite{Zygelman88}. While the problem has not been solved analytically, we will show that it can be solved numerically, to a desired precision. The radiative processes are driven by the interaction of the collision system with the radiation field. The direct change transfer is due to the transition between atomic (molecular) states due to the nuclear motion. Because the collision energy considered is low, typically only molecular states included are those which correspond to the initial $ A \prescript{1}{}\Sigma^+ $  and final $ X \prescript{1}{}\Sigma^+ $ channels.

Here we shall investigate the low energy collision process between the hydrogen atom and the hydrogen ion, in 2 dimension. We use fully quantum mechanical approach to calculate the cross section and the emission spectra of the reaction $ H(2\prescript{2}{}S) + H(1\prescript{1}{}S)\rightarrow H(1\prescript{1}{}S) + H(1\prescript{1}{}S) + \hbar\omega $, formed by the quenching of the excited $ H $ atom by the approaching $ H $ atom, in 2D dimensions. To my knowledge, there is no experimental results related to the 2D problem. There are results for the 3D case, listed in \cite{Zygelman88} and references there.

In this calculation, the atoms are confined in 2 dimensions, while the radiation is emitted or absorbed in all 3D space. Therefore the photon can be emitted in any direction and the potential felt by the incoming ion is the standard 3D Coulomb potential.

We will model this as a scattering problem, using a Born-Oppenheimer approximation and optical theorem. The Hamiltonian in this case will contain another term, namely the interaction of the radiation field with the electron

\subsection{Applications}

These type of reactions often occur in the astrophysical processes. They are generators of X-rays in in astrophysical processes. They also generate X-rays in stellar environments, atmospheric environments and the interstellar mediums. Spectroscopic analysis of the spectral lines allows us to probe such environments. \\
The other application of the charge transfer is examining the behaviour of low-temperature edge plasmas in laboratory fusion device.


\subsection*{Length gauge}

This is an interesting approach, which could potentially be used in calculation. In this thesis we would nit be using it.

The length gauge is a gauge transformation that replaces the vector potential for the field by the scalar potential for the quasi-static electric field \cite{LengthGauge3}.  In this gauge we take the Hamiltonian as: $ H = \mathbf{p}^2/2m + V(\mathbf{r})  + e\mathbf{E}\mathbf{r} $. The length gauge is convenient since both the Coulomb and the external fields are represented by the scalar potentials, which are additive. In the presence of the radiation field, the length gauge is obtained by the gauge transformation of the vector potential $ \mathbf{A} $, such that $ \mathbf{A} \rightarrow \mathbf{A} + \nabla \chi $ where $ \chi = - \mathbf{r} \cdot \mathbf{A} $. 

In the length gauge, the interaction Hamiltonian is:
\begin{equation}
\begin{split}
& H_{int} = -\sum_j{ \mathbf{r}\cdot\mathbf{E} } \\[.8em]
& \mathbf{E} = i\,\sum_{k\alpha}{\left(\frac{2\pi c k}{V}\right)^{1/2}\hat{\epsilon}_{k\alpha}\left(a_{k\alpha} - a^{\dagger}_{k\alpha}\right)}
\end{split}
\end{equation}
where $ a_{k\alpha} $ and $ a^{\dagger}_{k\alpha} $ are destruction and creation operators for the photon of momentum $ \hbar k $ and polarization $ \alpha $ respectively.

\section{Approximation}

Here we briefly derive the approximate solution, in order to get a 'feel' of what we are looking into.  We want the show the molecular potential of the $ H_2^+ $ ion.
We use the LCAO method and examine the Gerade and Ungerade potentials.

\subsection{Gerade and Ungergade Molecular Potentials}
In diatomic molecules terms Gerade (German for even) and Ungerade (German for odd)describe how the electronic wavefunction behaves under inversion symmetry, i.e. symmetry of the molecular orbital with respect to the inversio center.
If the place the internuclear axis passing through the center of the molecule, the inversion manifests with every coordinate point $ r $ with $ -r $ .
These terms re meaningful only for molecules or systems that possess a center of inversion, points such that $ r\,\rightarrow -r $ maps the molecule onto itself.
The definition:
\begin{itemize}
  \item Gerade(g): an orbital wavefunction $ \Psi(r) $ is gerade if it is unchanged by inversion: $ \Psi(-r) = \Psi(r) $. It has even parity. Example: $ 1s\sigma_g $

  \item Ungerade(u): an orbital wavefunction $ \Psi(r) $ is ungerade if it changes sign under inversion: $ \Psi(-r) = -\Psi(r)  $. It has odd parity. Example $ 2p\sigma_u $.
\end{itemize}

\subsection{Selection rules and consequences}
Electric dipole transitions: transitions between electronic states follow parity selection rule. Electric dipole operator is odd (Ungerade). Therefore allowed transitions must change parity: $ g \rightarrow u $. Transitions $ g \rightarrow g $ or $ u \rightarrow u $ are electric dipole forbidden (though allowed by higher multipoles or vibronic coupling).
\\
Orbital interactions: orbitals can only mix (form nonzero overlap/hybridize) if they belong to the same symmetry species, including parity. Thus g orbitals do not mix with u orbitals.

\subsection{LCAO}
The ground state expression gives amplitude for the 2-dimensional hydrogen atom \cite{YangXL}
\begin{equation}
\psi_{1s}({\bf r}) =\frac{4}{\sqrt{2 \pi}}e^{-2 \, r}
\end{equation}
where $ r = |{\bf r}|$. In the LCAO method, an approximate description of the 2D gerade molecule $H_{2} $, for internuclear separation ${\bf R}$ is given by
\begin{equation}
\psi_{g}({\bf r},{\bf R}) = C(R) \Bigl ( \psi_{1s}({\bf r}-{\bf R}/2 ) + \psi_{1s}({\bf r}+{\bf R}/2 ) \Bigr )
\end{equation}
where $C(R)$ is a normalization factor so that
\begin{equation}
\int d^2 {\bf r} \, |\psi_{g}({\bf r},{\bf R})|^2 = C^2(R) \Bigl (
2 + 2 \int d^2 {\bf r} \, \psi_{1s}({\bf r}) \psi_{1s}({\bf r}+{\bf R}) \Bigr ) =1
\end{equation}
or,
\begin{equation}
  C(R) = \frac{1}{\sqrt{2 (1+ S(R) )}} \text{   and} \quad S(R) \equiv \int d^2 {\bf r} \, \psi_{1s}({\bf r}) \psi_{1s}({\bf r}+{\bf R}) \label{eq:soverlap}
\end{equation}
We now evaluate the expectation values $ \langle V \rangle $ of
the potential
\begin{equation}
-\frac{1}{|{\bf r} - {\bf R}/2|} - \frac{1}{|{\bf r} + {\bf R}/2|}.
\end{equation}
Inserting the above wave amplitude to evaluate the expectation value, we get
\begin{equation}
\langle V \rangle_{g} =
-2 \int \frac{d^2 {\bf r}}{r} \, |\psi_{g}({\bf r}+{\bf R}/2 )|^2.
\end{equation}
We use the virial theorem to estimate the gerade ground state expectation value for the kinetic energy operator
\begin{equation}
\langle KE \rangle_{g} = \frac{1}{2} \langle {\bf r} \cdot {\bf \nabla} V \rangle
\end{equation}
and since
\begin{equation}
{\bf r} \cdot {\bf \nabla} V  = {\bf r}\cdot \frac{( {\bf r}- {\bf R}/2) }{ |{\bf r}- {\bf R}/2 |^{3} } + {\bf r}\cdot \frac{( {\bf r}+ {\bf R}/2) }{ |{\bf r}+ {\bf R}/2 |^{3} }
\end{equation}
we find
\begin{equation}
\langle KE \rangle_{g} =
 \int \frac{d^2 {\bf r}}{r} \, |\psi_{g}({\bf r}+{\bf R}/2 )|^2
+ \int \frac{d^2 {\bf r}}{r^3} \, ({\bf r}\cdot{\bf R}/2) \,
|\psi_{g}({\bf r}+{\bf R}/2 )|^2 .
\end{equation}
In the same manner, we obtain the BO energy for the ungerade BO state,
\begin{equation}
\psi_{u}({\bf r},{\bf R}) =\frac{1}{\sqrt{2 (1- S(R) )}}\int d^2 {\bf r} \, \Bigl ( \psi_{1s}({\bf r}-{\bf R}/2 ) - \psi_{1s}({\bf r}+{\bf R}/2 ) \Bigr ).
\end{equation}

In Table \ref{tab:lcaotab} we show the accurate vs LCAO values for the gerada state.

\begin{table}[ht]
  \caption{Accurate LCAO Energy as Function or R (internuclear distance) for the gerade state}
  \centering
  \label{tab:lcaotab}
  \begin{tabular}{ m{4em} m{4em} m{4em} }
    \hline
    R & E(R) + V(R) accurate  &  E(R) + V(R) LCAO \\ \hline \hline
    0.2 & -1.50328 & 1.11882 \\
 0.3 & -2.48639 & -0.436367 \\
 0.4 & -2.77548 & -1.14571 \\
 0.5 & -2.8382 & -1.51849 \\
 0.6 & -2.81481 & -1.72704 \\
 0.7 & -2.75759 & -1.84613 \\
 0.8 & -2.68852  & -1.91333 \\
 0.9 & -2.61749 -1.94936 \\
 1. & -2.54904 & -1.96642 \\
 1.1 & -2.4852 & -1.97198 \\
 1.5 & -2.28416 & -1.95159 \\
 2. & -2.13638 & -1.9237 \\
 2.5 & -2.0621 & -1.91727 \\
 3. & -2.02753 & -1.9224 \\
 3.5 & -2.01226 & -1.93066 \\
 4. & -2.00568  & -1.93856 \\
 4.5 & -2.00284 & -1.94523 \\
 5. & -2.00156 & -1.95066 \\
 6 & -2.00065 & -1.95882 \\
 7 & -2.00036 & -1.96462 \\
 8 & -2.00023 & -1.96898 \\
 9 & -2.00016 & -1.97238 \\
    10 & -2.00012 &  -1.97512 \\
    11 & -2.00009  & -1.97736 \\
    12 & -2.00006 & -1.97924 \\
  \hline
   \end{tabular}
\end{table}

Graphical comparison of the accurate vs LCAO potential for the gerade state. The accurate potential is in Red and the approximation is in blue.

\includegraphics{"Vg-Vs-LCAO-good.png"}

This can be easier observed in the enchanced plot. 

\includegraphics{"Vg-Vs-LCAO-2.png"}

Above graphs shows accurate gerade potentials (red dots), versus the ones obtained with the simplistic LCAO method described above.
We can observe that for $ R > 10 $ there is fair agreement; not so good for $ R < 10 $. This is a reasonable result as the LCAO including only the ground state orbital is a poor description in the molecular region.

We now show the same comparison for the ungeradate state in Table \ref{tab:lcaotab} we show the accurate vs LCAO values for the ungerada state.

\begin{table}[ht]
  \caption{Accurate and LCAO Energy as Function or R (internuclear distance) for the ungerade state}
  \centering
  \label{tab:lcaotab}
  \begin{tabular}{ m{4em} m{4em} m{4em} }
    \hline
    R & E(R) + V(R) accurate  &  E(R) + V(R) LCAO \\ \hline \hline
    0.2 & 4.15868  & 2.01244 \\
 0.3 & 2.42148 & 0.534181 \\
    0.4 & 1.46676  & -0.186921 \\
 0.5 & 0.798837 & -0.617186 \\
 0.6 & 0.273983 & -0.905274 \\
 0.7 & -0.153558 & -1.11241 \\
 0.8 & -0.502669 & -1.26834 \\
 0.9 & -0.786143 & -1.38934 \\
 1. & -1.01521 & -1.48521 \\
 1.1 & -1.1999 & -1.56225 \\
 1.5 & -1.64494 & -1.75286 \\
 2. & -1.86744 & -1.85654 \\
 2.5 & -1.95029 & -1.90027 \\
 3. & -1.98193 & -1.92094 \\
 3.5 & -1.99399 & -1.93255 \\
 4. & -1.99845 & -1.94038 \\
 4.5 & -2. & -1.94638 \\
 5. & -2.00046 & -1.95129 \\
 6 & -2.00049 & -1.95897 \\
 7 & -2.00034 & -1.96465 \\
 8 & -2.00023  & -1.96899 \\
 9 & -2.00016 & -1.97239 \\
 10 & -2.00012 & -1.97512 \\
 11 & -2.00009 & -1.97736 \\
 12 & -2.00007  & -1.97923 \\
  \hline
   \end{tabular}
\end{table}

Graphical comparison of the accurate vs LCAO potential for the ungerade state.

\includegraphics{"Vu-Vs-LCAO-good.png"}

Unlike for the gerade state, in the ungerade case the LCAO is quite a good approximation to the accurate potential. \\
The reason is the symmetries of the gerade and ungerade states and the overlap integral \eqref{eq:soverlap}.  
In the ungerade case we have:\\ 
$ \psi_{u}({\bf r},{\bf R}) \approx  \psi_{1s}({\bf r}-{\bf R}/2 ) - \psi_{1s}({\bf r}+{\bf R}/2 ) $ which makes $ 1 - S \approx 1 $.\\
It follows that the orbital are close to being orthogonal, which makes LCAO approximation more accurate.\\

In the gerade case, we have $ \psi_{g}({\bf r},{\bf R}) \approx  \psi_{1s}({\bf r}+{\bf R}/2 ) - \psi_{1s}({\bf r}+{\bf R}/2 ) $. In this case we have $ 1 + S > S $. In this case the LCAO orbits are much less orthogonal, so the overlapp error becomes much greater.

In the following figures we show the atomic orbitals for the gerade and ugerade case as a function of the internuclear distance R.

%\includepdf[pages=-,pagecommand=\subsubsection{Gerade potentials as function of the internuclear distance R},offset=0 -1.5cm]{Ger-Lcao-Graphs.pdf}
Gerade: 

\includegraphics[width=1\textwidth]{Ger-Lcao-Graphs.jpg}

Gerade 3D: 

\includegraphics[width=1\textwidth]{Ger-Lcao-3D-Graphs.jpg}

Ungerade: 

\includegraphics[width=1\textwidth]{UnGer-Lcao-Graphs.jpg}

Ungerade 3D: 

\includegraphics[width=1\textwidth]{UnGer-Lcao-3D-Graphs.jpg}

%\includepdf[pages=-,pagecommand=\subsubsection{Ungerade potentials as function of the internuclear distance R}, offset=0 -1.5cm]{UnGer-Lcao-Graphs.pdf}

\section{Accurate Solution Approach}

We start with the following Hamiltonians:
\begin{equation}\label{eqH1} 
\begin{split} 
  & H = -\frac{1}{2\mu}\nabla^2_{\mathbf{R}} + V_{g,u}(R)
\end{split} 
\end{equation}
where $ V_{g,u}(R) $ represents potential for the gerade (g) and ungerade state (u)

where $ \mu = \frac{m_p\,m_e}{m_p + m_e} $, is the reduced mass, $ \nabla_{\mathbf{R}} $ is the gradient operator for the relative nuclear motion. $ H_{el}(\mathbf{R},\mathbf{r}) $ is the fixed nuclei Hamiltonian for the electron, whose coordinate are labeled by $ \mathbf{r} $. $ H_{rad} $ is the Hamiltonian of the radiation field, and $ H_{int} $ is the radiation-matter coupling. Since we are dealing with the interaction of an atom with the EM radiation, it is common to use the length gauge.

From the physics of the system, we have two classes of solution, Gerade and Ungerade. Those correspond to the parity of the electron's wavefunction.

So following the \cite{Dalgarno1953} we have the $ |R| \rightarrow \infty $ asymptotic eigenstates of equation \eqref{hdia}, both gerade and ungerade:
\begin{equation}
\begin{split}
  & \Phi_g = \frac{1}{\sqrt{2}}\left(\Phi_a(\vec{r} - \frac{\vec{R}}{2}) + \Phi_b(\vec{r} + \frac{\vec{R}}{2})\right) \\
  & \Phi_u = \frac{1}{\sqrt{2}}\left(\Phi_a(\vec{r} - \frac{\vec{R}}{2}) - \Phi_b(\vec{r} + \frac{\vec{R}}{2})\right)
\end{split}
\end{equation}

So we now apply the scattering analysis for 2D, from the previous chapter, and assume that $ F_g(r) $ at asymptotic distances obeys the 2D conditions:
\begin{equation}
  F_g(R) \underset{R \rightarrow +\infty}{\sim} e^{ik\ x} + f_g(\theta)\frac{e^{ik\ x}}{\sqrt{R}}
\end{equation}
Now we use the 2D scattering amplitude from the previous chapter:
\begin{equation}\label{fgampl}
  f_g(\theta) = \sqrt{\frac{2}{Pi}}\sum_{m=0}^{\infty}{\epsilon_m \cos(m\theta)e^{i\delta_m^g}\sin\delta_m}
\end{equation}
As the $ f(\theta) $ is not directly observable, in 2D we look for the scattering length. And we get for the scattering length, in 2D:
\begin{equation}
\begin{split}
  & \theta_g = \int_{0}^{2Pi}{d\theta f_g(\theta)} = \frac{2}{Pi}\sum_{m,m'}\int_{0}^{2Pi}{\epsilon_m\epsilon_m' \cos(m\theta)\cos(m'\theta)e^{i\delta_{m}^g}e^{i\delta_{m'}^g}\sin\delta_m^g\delta_{m'}^g} = \\
  & = \frac{2}{Pi}\sum_{m}{\epsilon_m\sin^2\delta_m^g}
\end{split}
\end{equation}

The main approaches to solving these class of problems is to expand the wave function in the partial waves basis and use Born approximation.

\subsection{Using Partial Waves}

Using partial waves  we expand the incoming wave in the basis of spherical harmonics, and in the case of azimuthal symmetry to the basis of Legendre polynomials. This in effect mean that we decompose each wave into its constituent angular momentum components and solving using boundary conditions.

Born approximation \cite{GQuantum} treats a scattering potential as a perturbation to the incoming wave. In this case many partial waves contribute to scattering, so it is preferable to avoid angular momentum decomposition. The Born approximation is generally applicable either when the energy of the incoming particle(s) is high or when the scattering potential is very weak.

Again we write the time independent Schroedinger equation in 2D:
\begin{equation}
  -\frac{1}{2}\nabla^2\psi + V\psi = E\,\psi
\end{equation}

where m is the reduced mass, V is a short-range local potential, E is the relative energy.

Using polar coordinates, setting $ k^2 = 2mE $ , we have the equation
\begin{equation}
  \frac{1}{r}\frac{\partial}{\partial r}\left(r\frac{\partial\psi(r,\theta)}{\partial r}\right) + \frac{1}{r^2}\frac{\partial^2\psi(r,\theta)}{\partial \theta^2} + V(r,\theta) = E \psi(r,\theta)
\end{equation}

We expand the incoming wave in the partial wave basis as:
\begin{equation}
    e^{ikx} = e^{ikr\cos\theta} = \sum_{m=0}^{\infty}{\epsilon_m \cos(m\theta)J_m(kr)}
\end{equation}
where $\epsilon_m = 2 $ for $ m \ne 0 $ and $\epsilon_0 = 1 $.


We define a scattering length in 2 dimensions,, analogous to the scattering cross section in 3D, as:
\begin{equation}\label{scatterL}
  \lambda(\theta) = |\sqrt{\frac{i}{k}}f_m(\theta)|^2
\end{equation}

The equation \eqref{scatterL} represents the number of particles scattered between $ \theta $ and $ \theta + d\theta $ per second per unit incident flux.

We can express this function in the partial waves bases. Since $ V(r,\theta) $ contains no bound states we get for $ \mathbf{R} = \mathbf{R}(R,\theta) $:
\begin{equation}\label{GreenFka2}
\begin{split}
G^{+}(\mathbf{R}, \mathbf{R}') =  \frac{\pi\mu}{k_b}\sum_{l=0}^{\infty}{\sqrt{\frac{1}{2\pi}} P_l(\cos\theta) P_l^{*}(\cos\theta^{'})}\times \frac{f_l(k_bR_{<})g^{+}_l(k_bR_{>})}{RR'}
\end{split}
\end{equation}

where $ P_l(\cos\theta) $ are Legendre polynomials, also $ \theta = 0 $ part of the spherical harmonics:  $ P_l(\cos\theta) = Y_{l0}{\theta,0} $.

The $ f_l(R) $ is a regular solution of the homogeneous Schrodinger radial  equation for the 2D case: \cite{H2atom}:
\begin{equation}\label{eqRadial1}
\begin{split}
  & \frac{1}{R}\frac{d}{dR}\left(R \frac{d}{dR}\,f_m(R)\right) + \left\{k^2 - \frac{m^2}{R^2} - 2\mu\left[V_b(R) - V_b(\infty)\right] \right\}f_m(R) = 0 \\[.8em]
& k \equiv \sqrt{2\mu[E - \hbar\omega - V_b(\infty)}
\end{split}
\end{equation}
where:
The solution of the equation \eqref{eqRadial1} are Bessel functions $ J_m(kR) $ and $ N_m(kR) $ with the asymptotic behaviour:
\begin{equation}
\begin{split}
f_{m}(kR)\sim \sqrt{\frac{2}{\pi kR}}\cos\left[kR - \pi\frac{m+1/2}{2} \right],\,\,{R \rightarrow \infty}
\end{split}
\end{equation}
and $ g^{+}_l(kR) $ is irregular solution with the boundary condition at large $ R $.
\begin{equation}\label{remoteWave}
\begin{split}
g^{+}_{m}{(kR)} \sim \sqrt{\frac{2}{\pi kR}}\cos\left[kR - \pi\frac{m+1/2}{2} + \delta_m\right],\,\,{R \rightarrow \infty}
\end{split}
\end{equation}
and $  \delta_m $ is a phase shift.

The total wave function \eqref{ansatzWave1} must be symmetric under the interchange of the $ H $ nuclei, so that $ F_a(\mathbf{R}) = -F_a(-\mathbf{R}) $ and $ F_{k\alpha}(\mathbf{R}) =  F_{k\alpha}(-\mathbf{R}) $. Now we solve equations \eqref{Fk} and \eqref{Fka} in the distorted wave approximation. Then $ F_a(\mathbf{R}) $ is the solution of the \eqref{Fk} with the coupling term set to be zero and it can be expressed in the form:
\begin{equation}\label{Flong1}
F_a(\mathbf{R}) = \sum_{m=1}^{\infty}{P_m(\cos\theta)(2m+1)i^{l}\sqrt{\pi}\times \exp[i\delta_m]\frac{s_m(k_a\mathbf{R})}{k_a\mathbf{R} } }
\end{equation}

where
\begin{equation}\label{diffcrs1}
\frac{d\sigma}{d\omega} = \frac{8}{3}\left[\frac{\pi \mu}{k_a}\right]^2 \frac{1}{c^3}\omega^3 \sum_{J}{\left[J M_{J,J-1}^2(k_a,k_b) + (J+1)M_{J,J+1}(k_a,k_b)\right] }
\end{equation}

$ \Delta E $ is the energy of the transition at $ R = \infty $ and $ \omega_{max} $ is the maximum frequency of the emitted photon. Expression \eqref{crs1} is an equivalent expression to the Fermi's Golden Rule. Equation \eqref{diffcrs1} provides the spectrum of the emitted radiation, in addition to the scattering cross section.

% HERE \subsection{Phase Shift Calculation}

\subsection{Long Range Behavior of $E(R)$ }

Although the LCAO approximation given above correctly predicts the asymptotic energy
$ E(R) \rightarrow -2  $ as $ R \rightarrow \infty $, it does not accurately predic the long range behavior of $E(R).$ To that end we note that as $ R \rightarrow \infty $,
\begin{equation}
V({\bf r},{\bf R}) + 1/R \rightarrow -\frac{1}{r}-\frac{z}{R^2} + \frac{x^2-2 z^2}{2 R^3} + \dots
\end{equation}
where $z$ is the electronic coordinate along the direction ${\bf R}$, and we ignored terms that contain higner inverse powers of $R$. Thus the Hamiltonian
\begin{equation}
H \approx H_{0} -\frac{z}{R^2} + \frac{x^2-2 z^2}{2 R^3} + \dots
\end{equation}
where $H_{0}$ is the hamiltonian for the 2D hydrogen atom. Making use of perturbation theory,
\begin{equation}
E(R) \approx E_{0} + E^{(1)}(R) + E^{(2)}(R) + \cdots
\end{equation}
and
\begin{equation}
E^{(1)} = - \frac{\langle \psi_{1s} |z| \psi_{1s} \rangle }{R^2}
+  \frac{\langle \psi_{1s} |(x^2-2 z^2)| \psi_{1s} \rangle }{2 R^3}  =0
\end{equation}
The lowest non-vanishing contribution arises from second-order perturbation theory
\begin{equation}
  \begin{split}
& E^{(2)} = -\frac{C_{4}}{R^4}  \nonumber \\
& C_{4} = \sum_{n} \frac{|\langle \psi_{1s} |z | \psi_{n} \rangle |^{2} }{E_{1s} - E_{n} }
  \end{split}
\end{equation}
The polarization constant $C_{4} =0.082 $ was given in cite{YangXL}.

\section{Elastic Scattering}
In the PSS (perturbed stationary state) approximation, and after factoring out the center of mass coordinates, the system wave amplitude is given by the expression
\begin{equation}
\psi_{g}({\bf R}) \phi_{g}({\bf R},{\bf r}) + \psi_{u}({\bf R}) \phi_{u}({\bf R},{\bf r})
\label{0.00}
\end{equation}
where $ \phi_{u,g}({\bf R},{\bf r}) $ are, respectively, the gerade and ungerade electronic ground BO eigenstates
and $\psi_{g}({\bf R}), \psi_{u}({\bf R}) $ are the gerade and
un-gerade amplitudes for the relative motion of the atom and projectile. The amplitudes satisfy the following
\begin{equation}
-\frac{1}{2 \mu} {\bf \nabla^2} \psi_{u,g}({\bf R}) + V_{u,g}(R)
\psi_{u,g}({\bf R}) = E \, \psi_{u,g}({\bf R})
\label{0.01}
\end{equation}
where $ E = k^2/2\mu $ is the collision energy, and $ V_{u,g} $
are the Born-Oppenheimer potentials for the gerade and ungerade
electronic states respectively. Here $ {\bf \nabla}^2 $ is the two-dimensional Laplacian, and because $ V_{g,u}(R) $ are rotationally invariant, we can express the amplitudes as a partial wave expansion
\begin{equation}
\psi_{g,u}({\bf R}) = \sum_{m=-\infty}^{m=\infty} \, e^{i m \phi} F^{\, g,u}_{m}(R)
\end{equation}
where the radial functions $F_{m}(R)$ obey
\begin{equation}
F_{m}''(R) + \frac{F'_{m}}{R} - \frac{m^2}{R^2} F_{m}(R) + (k^2 - 2 \mu V_{g,u}(R)) F_{m}(R) =0
\end{equation}
where we suppressed the $g,u$ superscripts in the radial functions, and the prime notation denotes derivatives with respect to $R$.

In the asymptotic region as $ R \rightarrow \infty $ and $ V_{g,u} \rightarrow 0$,
Eq. (xx) tends to the Bessel equation for integer order, and the radial solutions tend to
\begin{equation}
F_{m}(R)\rightarrow a_{m} \, J_{|m|}(k R) +
b_{m} H^{(1)}_{|m|}(k R)
\end{equation}
where $J_{m}(k R)$ is the regular Bessel function
and $ H^{1}_{m}(k R) $ is a Hankel function of the first kind,
and $a_{m},b_{m} $ are constants. We impose the boundary condition
\begin{equation}\label{psiLR}
\psi({\bf R}) \rightarrow e^{i \, k R \, \cos\phi}
+ f(\phi) \, \frac{e^{i \, k  R}}{\sqrt{R}} .
\end{equation}
Using the identity ,
\begin{equation}
e^{i k R \cos\phi} = \sum_{m=-\infty}^{m=\infty} \, -i^{-|m|} J_{|m|}( k R) e^{i m \phi}
\end{equation}
\begin{equation}
e^{i k R \cos\phi} = \sum_{m=-\infty}^{m=\infty} \, -i^{-|m|} J_{|m|}( k R) e^{i m \phi}
\end{equation}
we set $ a_{m} = -i^{-|m|}$ and inserting into \eqref{psiLR} we get
\begin{equation}
  \begin{split}
    \psi({\bf R}) \rightarrow e^{i \, k R \, \cos\phi}+ \nonumber \frac{e^{ i k R}}{\sqrt{R}} \, \sqrt{\frac{2}{\pi k}} \sum_{m}
(-1)^{(m+1)} e^{- i \frac{\pi}{4}} e^{i m \phi} \,b_{m}
  \end{split}
\end{equation}
where we used the asymptotic form
\begin{equation}
H^{(1)}_{m} \rightarrow \sqrt{\frac{2}{\pi k R}} \,
  e^{\left (i \, (k \, R -\frac{m \pi}{2} -\frac{\pi}{4}) \right )}.
\end{equation}
We can solve for the coefficients $b_{|m|}$ by matching the radial logarithmic derivative
\begin{equation}
y_{m} \equiv \frac{F'_{m}}{F_{m}}
\end{equation}
evaluated at a cutof distance $R_{c}$ so that the ratio
\begin{equation}
i^{-|m|} \, \frac{J'_{|m|} - y_{m} \, J_{|m|} }{ \, H'^{(1)}_{|m|} - y_{m} \, H^{(1)}_{|m|} }
\end{equation}
tends to a constant value $b_{|m|} $ at $ R>R_{c}.$

Using the expression
\begin{equation}
  {\bf j} = Re \Bigl [\frac{-i}{\mu} \, \psi^{*}({\bf R}){\nabla }\psi^{*}({\bf R})  \Bigr ]
\end{equation}
where ${\bf \nabla}$ is the two-dimensional gradient operator for the probability current, we obtain for the incident current
along the $z-$ direction
\begin{equation}
j_{inc} = \frac{ k}{\mu}
\end{equation}
and
for the leading order term of outgoing radial current
\begin{equation}
j_{r} = \frac{k}{R \mu } f^{*}(\phi)f(\phi)
\end{equation}
\begin{equation}
j_{r} = \frac{k}{R \mu } f^{*}(\phi)f(\phi)
\end{equation}
Defining the cross section as the ratio of particle flux scattered into the region $ R \, d \phi $ centered at angle $\phi$ with the incident current $j_{inc}$, we get
\begin{equation}
d\sigma   \equiv   |f(\phi)|^2 \, d\phi
\end{equation}
Comparing (xx) with (xx)  we find that
\begin{equation}
f(\phi) =
\sqrt{\frac{2}{\pi k}} \sum_{m}
(-1)^{(m+1)} e^{- i \frac{\pi}{4}}\,e^{i m \phi} \,b_{|m|}
\end{equation}
Thus
\begin{equation}
\sigma= \int_{0}^{2\pi} d\phi \, \frac{d\sigma}{d \phi} =
\int_{0}^{2\pi} d\phi |f(\phi)|^2 =
\frac{4}{k} \sum_{m} |b_{|m|}|^2
\end{equation}
Alternatively, the constant $ b_{m} $ is sometimes expressed as a phase shift.

\section{Resonant Charge Transfer Cross Sections}

Consider the amplitude given by expression (xx)
\begin{equation}
F({\bf R},{\bf r}) = \psi_{g}({\bf R}) \phi_{g}({\bf R},{\bf r}) + \psi_{u}({\bf R}) \phi_{u}({\bf R},{\bf r})
\end{equation}
In the limit as $ R \rightarrow \infty$,
\begin{equation}
F \rightarrow \psi_{g}({\bf R}) \frac{1}{\sqrt{2}} (\phi_{a} + \phi_{b} ) +
\psi_{u}({\bf R}) \frac{1}{\sqrt{2}} (\phi_{a} - \phi_{b} )
\end{equation}
where
\begin{equation}
\phi_{a} \equiv \psi_{1s}({\bf r}-{\bf R}/2)
\quad \phi_{b} \equiv \psi_{1s}({\bf r}+{\bf R}/2)
\end{equation}
Now imposing the boundary conditions
\begin{equation}
\psi_{g,u}({\bf R}) \rightarrow
e^{i k R \cos\phi} + f_{g,u}(\phi) \frac{e^{i k R}}{\sqrt{R}}
\end{equation} 
we get
we get
\begin{equation}
F \rightarrow \sqrt{2}e^{i k R \cos\phi}\phi_{a}
+ \frac{1}{\sqrt{2}} (f_{g}(\phi)+f_{u}(\phi))  \phi_{a}  + \frac{1}{\sqrt{2}} (f_{g}(\phi)-f_{u}(\phi))  \phi_{b}
\end{equation} 
So the ratio of flux coming into the $\phi_{a}$ channel to that going into the $\phi_{b} $ channel (i.e. resonant charge transfer) is given by
\begin{equation}
\sigma_{RCT}= \frac{1}{4} \, \int_{0}^{2 \pi} |f_{g}(\phi)-f_{u}(\phi) |^2 = \frac{1}{k} \sum_{m} \Bigl |b^{(g)}_{|m|} - b^{(u)}_{|m|} \Bigr |^2
\end{equation}
where $ b^{(g,u)}_{|m|} $ are the elastic
scattering coefficients for the gerade and ungerade potentials, respectively.

HERE


\section{Solution Strategies}
The main approaches to solving problems is to expand the wave function in the partial waves basis and use Born approximation.

Using partial waves  we expand the incoming wave in the basis of spherical harmonics, and in the case of azimuthal symmetry to the basis of Legendre Polynomials. This in effect mean that we decompose each wave into its constituent angular momentum components and solving using boundary conditions.

Born approximation \cite{GQuantum} treats a scattering potential as a perturbation to the incoming wave. In this case many partial waves contribute to scattering, so it is preferable to avoid angular momentum decomposition. The Born approximation is generally applicable either when the energy of the incoming particle(s) is high or when the scattering potential is very weak. 

\section{Phase Shift Calculation}

We follow the approach in \cite{Dalgarno1953} and in the following expression for the scattering length we compute phase shifts for both gerade and ungerade potentials.

So for the purely elastic single-channel 2D cross section, we have:
\begin{equation}
\begin{split}
  \lambda = \frac{4}{k}\sum_{m=0}^{\infty}{sin^2(\delta_m^{+}-\delta_m^{-})}
\end{split}
\end{equation}

It is now fairly straightforward to compute the scattering length. We compute the scattering length in the Wolfram Mathematica code listed in \cite{AppendixE}.

The following is the plot of the scattering length in a.u. vs temperature in Kelvin.

\includegraphics{CrossSection1.png}

\subsection{Optical Potential Method, Dipole Moment and Einstein A Coefficient}

The approximation that does not require the integration over the total spectrum is the optical potential method. 

Again it is assumed that the scattering takes place at low energies, so that the system can be described by a wave function $ \Psi(x,t) $ which is itself a solution of a Schrodinger equation $ H\ket{\Psi(t)} = i\hbar\frac{\partial}{\partial t}\ket{\Psi(t)} $ where $ H $ is a Hamiltonian $ H = T + V $ .
Then we borrow a concept from the optics, where there is a concept of a Complex Refractive Index $ n $. The real part of $ n $ describes how the light is transmitted and the imaginary part describes how the light is absorbed. So it the imaginary part that describes the effect of the scattering. We then express the potential $ V = U + iW $  and the Schrodinger equation becomes $ \left[H + U + iW\right]\ket{\Psi(t)} = i\hbar\frac{\partial}{\partial t}\ket{\Psi(t)} $. From this one gets the scattering cross section:
\begin{equation}
\sigma = \frac{k}{E}\int{d^3R\left[-W(R)\right]\left| \psi(R) \right|}
\end{equation}

To derive it here we insert the equation \eqref{FkaInt1} for the amplitude $ F_{k\alpha}(\mathbf{R}) $ into the equation \eqref{Fk} to obtain the equation for the amplitude $ F_a(\mathbf{R}) $
\begin{equation}\label{opt1}
\begin{split}
\left[-\frac{1}{2\mu}\nabla_R^2 + V_a(R) - E \right]F_a(\mathbf{R}) = \sum_{k\alpha}{d^2R'G^{+}(\mathbf{R},\mathbf{R}^{'})U_{k\alpha}^{\dagger}(\mathbf{R}^{'})U_{k\alpha}(\mathbf{R})F_a(\mathbf{R}^{'})}
\end{split}
\end{equation}

The right hand of \eqref{opt1} contains a complex, non-local potential
\begin{equation}\label{optV1}
V(\mathbf{R},\mathbf{R}^{'}) = \sum_{k\alpha}{G^{+}(\mathbf{R},\mathbf{R}^{'})U_{k\alpha}^{\dagger}(\mathbf{R}^{'})U_{k\alpha}(\mathbf{R}) }
\end{equation}

that arises because of the interaction of the electron with the vacuum. Now the real part of this potential induces the shift in the eigenvalue $ V_a(R) $. Since the coupling on an electron with the radiation is weak, we can ignore it and consider only the imaginary part. 

The imaginary part of $ V(\mathbf{R},\mathbf{R}^{'}) $ is an absorptive potential, representing a process where electron in excited state emits a photon and decays to a ground state. This potential is non-local, as we take is for $ R = \infty $. In the optical potential approximation, we take replace potential by the local one, whose range is limited to the scattering area. This potential is essentially classical.

Because the term $ U_{k\alpha}^{\dagger}(\mathbf{R}^{'})U_{k\alpha}(\mathbf{R}) $ appearing in equation \eqref{optV1} is real, the optical potential is proportional to the imaginary part of the retarded Green's function, which is expressed as:
\begin{equation}\label{OptGreen1}
\begin{split}
\text{Im} G^{+}(\mathbf{R},\mathbf{R}) = \pi\sum_{l=0}^{\infty}{\cos(l\, \theta)\cos(l\,\theta^{'})\int_0^{\infty}{dk\,\delta\left[\frac{k^2}{2\mu} - \frac{k^2_a}{2\mu} + \hbar\omega - \Delta E \right]\frac{f_l(kR)f_l(kR^{'})}{R\,R^{'}} }  }
\end{split}
\end{equation}

