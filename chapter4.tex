\chapter{Radiative Processes}
\label{chap:RRP}
\section{Description}

As stated we will focus on a charge transfer processes in low energies, i.e. low temperatures. We consider a positively charged ion that approaches a neutral atom. As the internuclear distance R becomes smaller, there exist a non-zero probability that the electron will escape the atom and attach itself to the incoming ion. In this manner a charge transfer happens.

There are two classes of Charge Transfers,

The analysis of the radiative processes is related to the analysis of the general scattering process. So we have briefly touched scattering in the previous section.l

Typically for the scattering processes at this level, we can distinguish between the radiative and non-radiative processes. 
Following \cite{ZygelmanCT} we consider a positively charged Hydrogen ion approaching a neutral Hydrogen atom. As the values or R (internuclear distance) becomes smaller, there increases a probability that the electron will escape one atom and attach itself to another.\\
When this happens a charge transfer has occurred. \\
There are two classes of transfer, endoergic and exothermic .\\
In the endoergic case the binding energy of the electron on the ion is less than that of the atom. In this case,in order to satisfy the energy conservation requirements, an incoming ion needs to supply the additional energy. However if the ion approaches slowly, there is no extra supply of energy. So when an additional energy is not available, charge transfer can only occur if the final electron binding energy is greater than or equal to its binding energy on the atom.\\
If the energy is equal, the reaction is called resonant charge transfer.\\
if not the reaction will processed with the possibility of the emission of a photon.
If radiation is emitted, we call the reaction radiative charge transfer/association and, if not, direct charge transfer.

Radiative processes involve emission of a photon and they typically mean Radiative Charge Transfer: \\
$ A(n+1\prescript{1}{}S) + B \longleftrightarrow A(n\prescript{1}{}S) + B + \hbar\omega $ \\
and Radiative Association: \\
$ A(n+1\prescript{1}{}S) + B \longleftrightarrow A(n\prescript{1}{}S)B + \hbar\omega $ .\\
For example, the simplest case of the radiative process is the collision of the two hydrogen atoms, in the process $ H + H^+ \rightarrow H^+ + H + \hbar\omega $.\\

While the non-radiative process, with no photon emission, is a Charge Transfer: \\
$ A(n+1\prescript{1}{}S) + B(n\prescript{1}{}S) \longleftrightarrow  A(n\prescript{1}{}S) + B(n+1\prescript{1}{}S) $ .\\

Radiative Quenching is an interaction between two atoms (or molecules), one being in excited state and the other in 'normal' ground state. During this process the excited atom emits a photon and drops to a ground state.
The simplest such case is the collision of the two hydrogen molecules, in the process $ H(2\,{}^2\!S) + H(1\,{}^2\!S) \rightarrow H(1\,{}^2\!S) + H(1\,{}^2\!S) + \hbar\omega $.

There are several theoretical approaches and treatments of this problem. \cite{RadQuench1}, \cite{RadQuench2}, \cite{Zygelman88}. 
\subsubsection{Classical Treatments}

\subsubsection{COB model}
The classical 'over the barrier model' \cite{BransdenMcDowell1992} treats an electron as being inside a potential well. The well's potential is described by the function
\begin{equation}
  V(r) = -\frac{q}{x} - \frac{1}{\lvert R - r \rvert}
\end{equation}
where $ r $ is the distance of the bound electron from the incoming nucleus. $ R $ is the distance between the nuclei, and $ q $ is the charge of the ion. 

Potential has it maximum as $ V_{max} = -\left(\sqrt{q} + 1\right)^2/R $. The electron attaches itself to an incoming ion if its binding energy, perturbed by the electron–ion Coulomb energy $ -1/2 - q/R $ becomes greater that $ V_{max} $.

Assuming that the electron's binding energy is $ E_n = -q^2/2n^2 $ where $ n $ is a principal quantum number, energy conservation requires that 
\begin{equation}
  -\frac{1}{2} - \frac{q}{R} = E_n - \frac{1}{R}
\end{equation}
Together these relations predict the largest value of $ R $ where the electron is transferred to an incoming ion with an energy $ E_n $ to be:
\begin{equation}\label{COBRT}
  R_n = \frac{2(q-1)}{q^2/n^2-1}
\end{equation}
Assuming the transfer probability of $ 1/2 $ this model predicts the total charge transfer cross section to be:
\begin{equation}
  \sigma_{CT} = \frac{\pi}{2}R_{n^*}^2
\end{equation}
where $ n^* $ is the value of $ n $ in equation \eqref{COBRT} where the equality holds. We shall compare this value with the value obtained by our method. But generally speaking this method fails to give the correct values for the very low energy processes \cite{osti_6533174}.

\subsubsection{Langevin Orbiting}
Since the preceding model fails to give the correct prediction a new, still classical model was suggested by P. Langevin. In this model, the incoming ion introduces the energy shift of the electron's ground state by 
\begin{equation}
  \Delta E = -\frac{1}{2}\alpha|E|^2  
 \end{equation}
 where $ \alpha $ is called the polarizability of the atom and $ E = q/R^2 $. The mutually attractive potential has the form :
\begin{equation}
  V_{pol}(R) = -\frac{C_4}{R^4}\,\,\,\,\,C_4 = \frac{\alpha}{2}q^2
\end{equation}
in addition to the centripetal potential
\begin{equation}
  V_L(R) = \frac{L(L+1)}{2\mu R^2}
\end{equation}
where $ L = \mu v b $ is an angular momentum and $ b $ is an impact parameter. For a given $ b $, the effective potential, the sum of $ V_{pol} $ and $ V_L $ possesses a local maximum $ V_{max}(b) $ at critical value $ R_c $. 
At the end this model predicts the total charge transfer cross section to be:
\begin{equation}
  \sigma_{CT} = 2\pi \int_0^{b_c}{db\, b} = \pi b_c^2 = \frac{\pi}{v}\sqrt{\frac{8 C_4}{\mu}} = \frac{2\pi q}{v}\sqrt{\frac{\alpha}{\mu}}
\end{equation}
\subsubsection{Landau–Zener–Rosen Theories}
This is a more advanced method which takes into account quantum nature of the particles. We shall skip the explanation, just to note that this model predicts the cross section to be:
\begin{equation}
\begin{split}
  & \sigma_{CT} = 2\pi \int_0^{R_c}{b P(b) db} \\[.8em]
  & P(b) = \frac{1}{2}sech^2\left(\frac{\pi \Delta E(R_c)}{2\lambda v(b)}\right)
\end{split}
\end{equation}

\subsection{Quantum Mechanical Treatment, MOCC Approach}

The Molecular Orbital Approach (MOCC) assumes that the total system amplitude is expressed as a sum of electronic
eigenstates, whose expansion coefficients describe the motion of the nuclear coordinate R. There are two versions of
this theory. In the first, the nuclear motion is treated classically, but the electronic degrees of freedom are treated quantum mechanically. This approach, the method of perturbed stationary states (PSS), was first advocated by Massey and Smith in 1932.
The quantum version of this theory treats both the nuclear motion and the electronic degrees of freedom quantum mechanically. It was developed by David Bates and coworkers in the 1950s. It was not much used since it required a sizeable computing power. So only in the 80s the fully quantum MOCC computation becomes available. \cite{ZygelmanCT}.

In the fully quantum expressions of the PSS equations we get for the wavefunction

\begin{equation}
  \Psi(\mathbf{r},\mathbf{R}) = \sum_n^N{\phi_n(\mathbf{r},\mathbf{R})F_n(R)}
\end{equation}
where $ F_n(R) $ are quantal amplitudes that predict the prob- ability for the system to occupy the n-th electronic states
at R. Index $ N $ basically includes all the states that are energetically open.
The states $ \phi_n $ form a complete set 
\begin{equation}
  \sum_n{\phi_n^{\dag}(\mathbf{r},\mathbf{R})F_n(R)\phi_n(\mathbf{r'},\mathbf{R})F_n(R)} = \delta(\mathbf{r} - \mathbf{r'})
\end{equation}

While the problem has not been solved analytically, we will show that it can be solved numerically, to a desired precision. The radiative processes are driven by the interaction of the collision system with the radiation field. The direct change transfer is due to the transition between atomic (molecular) states due to the nuclear motion. Because the collision energy considered is low, typically only molecular states included are those which correspond to the initial $ A \prescript{1}{}\Sigma^+ $  and final $ X \prescript{1}{}\Sigma^+ $ channels.

Here we shall investigate the low energy collision process between the hydrogen atom and the hydrogen ion, in 2 dimension. We use fully quantum mechanical approach to calculate the cross section and the emission spectra of the reaction $ H(2\prescript{2}{}S) + H(1\prescript{1}{}S)\rightarrow H(1\prescript{1}{}S) + H(1\prescript{1}{}S) + \hbar\omega $, formed by the quenching of the excited $ H $ atom by the approaching $ H $ atom, in 2D dimensions. To my knowledge, there is no experimental results related to the 2D problem. There are results for the 3D case, listed in \cite{Zygelman88} and references there.

In this calculation, the atoms are confined in 2 dimensions, while the radiation is emitted or absorbed in all 3D space. Therefore the photon can be emitted in any direction and the potential felt by the incoming ion is the standard 3D Coulomb potential.

We will model this as a scattering problem, using a Born-Oppenheimer approximation and optical theorem. The Hamiltonian in this case will contain another term, namely the interaction of the radiation field with the electron

\subsection{Selection rules and consequences}
Electric dipole transitions: transitions between electronic states follow parity selection rule. Electric dipole operator is odd (Ungerade). Therefore allowed transitions must change parity: $ g \rightarrow u $. Transitions $ g \rightarrow g $ or $ u \rightarrow u $ are electric dipole forbidden (though allowed by higher multipoles or vibronic coupling).
\\
Orbital interactions: orbitals can only mix (form nonzero overlap/hybridize) if they belong to the same symmetry species, including parity. Thus g orbitals do not mix with u orbitals.

\subsection*{Length gauge}

This is an interesting approach, which could potentially be used in calculation. In this thesis we would not be using it.

The length gauge is a gauge transformation that replaces the vector potential for the field by the scalar potential for the quasi-static electric field \cite{LengthGauge3}.  In this gauge we take the Hamiltonian as: $ H = \mathbf{p}^2/2m + V(\mathbf{r})  + e\mathbf{E}\mathbf{r} $. The length gauge is convenient since both the Coulomb and the external fields are represented by the scalar potentials, which are additive. In the presence of the radiation field, the length gauge is obtained by the gauge transformation of the vector potential $ \mathbf{A} $, such that $ \mathbf{A} \rightarrow \mathbf{A} + \nabla \chi $ where $ \chi = - \mathbf{r} \cdot \mathbf{A} $. 

In the length gauge, the interaction Hamiltonian is:
\begin{equation}
\begin{split}
& H_{int} = -\sum_j{ \mathbf{r}\cdot\mathbf{E} } \\[.8em]
& \mathbf{E} = i\,\sum_{k\alpha}{\left(\frac{2\pi c k}{V}\right)^{1/2}\hat{\epsilon}_{k\alpha}\left(a_{k\alpha} - a^{\dagger}_{k\alpha}\right)}
\end{split}
\end{equation}
where $ a_{k\alpha} $ and $ a^{\dagger}_{k\alpha} $ are destruction and creation operators for the photon of momentum $ \hbar k $ and polarization $ \alpha $ respectively.

\subsection{Radiative Association}

Considering the radiative association process 
\begin{equation}
  H(1S) + H^+ = H_2^+(1s\sigma_g) + \hbar\omega
\end{equation}
where $ \hbar\omega $ is the energy of the emitted photon given by the expression \eqref{hbaromega1}. 
ts cross section for the radiative association process is given by:
\begin{equation}
  \begin{split}
    &  \sigma_{RA} = \sum_J{\sum_n{\frac{8}{3}\frac{\pi^2\omega_{nJ}^3}{c^3k^2}\left[(J+1)M_{J+1,J}^2(k,n)+M_{J-1,J}^2(k,n)\right] }}\\[.8em]
    & M_{J,J'}(k,n) = \int_{0}^{\infty}{dR\,f_J(kR)D(R)\phi_{J'n}(R) }
  \end{split}
\end{equation}
where $ D(R) $ is a transition dipole moment between $ X^2\Sigma^+ $ state and $ A^2\Sigma^+ $ states of the $ H_2^+ $ molecular ion. $ \phi_{J'n}(R)$ is is a rho-vibrational eigenstate of the $ X^2\Sigma^+ $ ground state, with energy eigenvalue $ \epsilon_{nJ} $ and is characterized by the angular and vibrational momentum quantum numbers $ J $, $ n $ respectively.
$ f_J(kR) $ is the wavefunction that satisfies the radial Schrodinger equation
\begin{equation}\label{diffR}
  f_J^{''}(kR) - \frac{J(J+1)}{R^2}f_J(kR) + 2\mu V_A(R)f_J(kR) + k^2f_j(kR) = 0
\end{equation}
where $ V_A(R) $ is the Born-Oppenheimer (BO) energy of the excited $ A^2\Sigma^+ $ state, $ \mu $ is the reduced mass of the collision system and $ k $ is the wavenumber for the incident collision partners in that channel. It has the asymptotic form
\begin{equation}\label{fjRZ}
  f_J(kR) \rightarrow \sqrt{\frac{2\mu}{\pi k}\sin(kR -\frac{J\pi}{2} + \delta_J } 
\end{equation}
where $ \delta_J $ is phase shift, as $ R \rightarrow \infty $. The energy of the emitted photon is given by:
\begin{equation}\label{hbaromega1}
\hbar\omega_{nJ} = \frac{\hbar k^2}{2\mu} + V_A(\infty) - \epsilon_(nJ) - V_X(\infty)
\end{equation}

\subsection{Radiative Charge Transfer}

Considering the radiative charge transfer process
\begin{equation}
  H(1S) + H^+ = H^+ H(1S) + \hbar\omega
\end{equation}
where $ \hbar\omega $ is the energy of the emitted photon given by the expression \eqref{hbaromega2}.
\begin{equation}\label{sumZZ}
\begin{split}
& sigma_{CT} = \int_0^{\omega_max}{d\omega\,\frac{d\sigma}{d\omega}},\\[.8em]
& \frac{d\sigma}{d\omega} = \sum_J{\frac{8}{3}\frac{\pi^2\omega_{nJ}^3}{c^3k^2}\left[(J+1)M_{J+1,J}^2(k,k')+J\,M_{J-1,J}^2(k,k')\right] }
\end{split}
\end{equation}
where
\begin{equation}\label{mjjz}
  M_{J,J'}(k,n) = \int_{0}^{\infty}{dR\,f_J(kR)D(R)f_J'(k'R }
\end{equation}
Here $ f_J(kR) $ is a solution to \eqref{diffR} and $ f_J^{'}(k^{'}R) $ obeys the corresponding equation for the $ X^2\Sigma^+ $ exit channel with wavenumber and partial wave $ k' $, $ J'$ respectively.
The radial wavefunctions are normalized as in \eqref{fjRZ} and
\begin{equation}\label{hbaromega2}
\begin{split}
  & \hbar\omega = \frac{\hbar k^2}{2\mu} - \frac{\hbar k^{'2}}{2\mu} + \Delta E \\[.8em]
  & \Delta E = V_A(\infty) - V_X(\infty)
\end{split}
\end{equation}
From the \eqref{hbaromega2} the maximum angular frequency $ \omega_max $ is
\begin{equation}
\hbar\omega_{max} = \frac{\hbar k^2}{2\mu} +  \Delta E
\end{equation}
We avoid evaluating the sum \eqref{sumZZ}. Instead we choose to compute an upper bound on the charge transfer cross section.
From the equation above we get for the frequency of the emitted photon, during an RCT transition, to be
\begin{equation}
\hbar\omega = \frac{\hbar k^2}{2\mu} - \frac{\hbar k^{'2}}{2\mu} + V_A(\infty) - V_X(\infty) = \frac{\hbar k^2}{2\mu} - E' + V_A(\infty) - V_X(\infty)
\end{equation}
Now from the equation above we can see that to achieve the maximum energy of the emitted photon, $ \hbar\omega_{max} $ we need to have $ E' = 0 $.
The maximum value of $ E' $ is achieved when $ \hbar\omega = 0 $ therefore:
\begin{equation}
  E_max^{'} = \frac{\hbar k^2}{2\mu} + V_A(\infty) - V_X(\infty)
\end{equation}
Now the \eqref{sumZZ} can be written as:
\begin{equation}
  \sigma = \frac{8}{3}\frac{\pi^2}{c^3k^2}\int_0^{E_{max}^{'}}{dE'\,\omega^3(E')\left[(J+1)M_{J,J+1}^2(k,E')+J\,M_{J,J-1}^2(k,E')\right]}
\end{equation}
where we have the inequality
\begin{equation}\label{sigmalessthan}
  \sigma_{CT} < \frac{8}{3}\frac{\pi^2\omega^3(E')_{max}}{c^3k^2}\int_0^{E_{max}^{'}}{dE'\,\left[(J+1)M_{J,J+1}^2(k,E')+J\,M_{J,J-1}^2(k,E')\right]}
\end{equation}
Now we can replace the expression for $ M_{J+1,J}^2(k,E') $ in \eqref{sigmalessthan} with the expression in \eqref{mjjz} and squaring it,  we get the integral
\begin{equation}
  \int_0^{\infty}{dE'\,JM_{J,J-1}^2(k,E')} = \int_0^{\infty}{dE'\int_{0}^{\infty}{dR\,f_J(kR)D(R)f_{J-1}(k'R}\int_{0}^{\infty}{dR'\,f_J(kR')D(R)f_{J-1}(k'R' }   }
\end{equation}
so the inequality becomes
\begin{equation}
  \int_0^{\infty}{dE'\,JM_{J,J-1}^2(k,E')} < \sum_{E'}{JM_{J,J-1}^2(k,E')} = J\int_{0}^{\infty}{dR\,f_J^2(kR)D^2(R)}
\end{equation}
and we get for the upper bound
\begin{equation}
  \sigma_{CT} < \frac{8}{3}\frac{\pi^2\omega^3(E')_{max}}{c^3k^2}\sum_{J}{(2J+1)\int_{0}^{\infty}{dR\,f_J^2(kR)D^2(R)}}
\end{equation}

\subsection{Optical Potential Approach}

The alternative way to compute the cross section is to use the local optical potential method \eqref{Zygelman88}.  The loss of flux from the entrance channel equals the radiative probability, which is convert to a radiative cross section.
In this method the collision system in the incoming $ A^2\Sigma^{+} $ state experiences imaginary (absorptive) potential $ iA(R)/2 $ added to the entrance-channel BO potential.
So we define a complex potential 
\begin{equation}
\begin{split}
  & V(R) = V_{BO}(R) + V_{OPT}(R) = V_{BO}(R) + i\frac{A(R)}{2} \\[.8em]
  & A(R) = \frac{4}{3c^2}D^(R)[V_A(R) - V_X(R)]
\end{split}
\end{equation}
$ A(R) $ is the Einstein-A coefficient. We shall use a 2D version of the optical theorem to compute the radiative charge transfer in 2D case in the next chapter.
The next step is to solve the usual radial scattering equation, but with $ V(R) $ complex. The resulting S matrix becomes non-unitary, $ |S| < 1 $ and the “missing probability” is the radiative loss.

The cross section for radiative quenching is given by
\begin{equation}
  \sigma = \frac{\pi}{k^2}\sum_J{(2J+1)(1 - e^{-4\eta_j }}
\end{equation}

Local Optical Potential is generally applicable when the following cases are satisfied, which aligns well with our case.
\begin{itemize}
  \item Transition happens over a region small compared to relevant wavelengths
  \item Emission is relatively weak compared to the collision energies
  \item It is accurate enough for the computation of total probability
\end{itemize}

