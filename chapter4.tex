\chapter{Resonant Charge Transfer}
\label{chap:RxRP}
\section{Description}

\chapter{Radiative Charge Transfer}

Considering the radiative charge transfer process
\begin{equation}
  H(1S) + H^+ = H^+ H(1S) + \hbar\omega
\end{equation}
where $ \hbar\omega $ is the energy of the emitted photon given by the expression \eqref{hbaromega2}.
\begin{equation}\label{sumZZ}
\begin{split}
& sigma_{CT} = \int_0^{\omega_max}{d\omega\,\frac{d\sigma}{d\omega}},\\[.8em]
& \frac{d\sigma}{d\omega} = \sum_J{\frac{8}{3}\frac{\pi^2\omega_{nJ}^3}{c^3k^2}\left[(J+1)M_{J+1,J}^2(k,k')+J\,M_{J-1,J}^2(k,k')\right] }
\end{split}
\end{equation}
where
\begin{equation}\label{mjjz}
  M_{J,J'}(k,n) = \int_{0}^{\infty}{dR\,f_J(kR)D(R)f_J'(k'R }
\end{equation}
Here $ f_J(kR) $ is a solution to \eqref{diffR} and $ f_J^{'}(k^{'}R) $ obeys the corresponding equation for the $ X^2\Sigma^+ $ exit channel with wavenumber and partial wave $ k' $, $ J'$ respectively.
The radial wavefunctions are normalized as in \eqref{fjRZ} and
\begin{equation}\label{hbaromega2}
\begin{split}
  & \hbar\omega = \frac{\hbar k^2}{2\mu} - \frac{\hbar k^{'2}}{2\mu} + \Delta E \\[.8em]
  & \Delta E = V_A(\infty) - V_X(\infty)
\end{split}
\end{equation}
From the \eqref{hbaromega2} the maximum angular frequency $ \omega_max $ is
\begin{equation}
\hbar\omega_{max} = \frac{\hbar k^2}{2\mu} +  \Delta E
\end{equation}
We avoid evaluating the sum \eqref{sumZZ}. Instead we choose to compute an upper bound on the charge transfer cross section.
From the equation above we get for the frequency of the emitted photon, during an RCT transition, to be
\begin{equation}
\hbar\omega = \frac{\hbar k^2}{2\mu} - \frac{\hbar k^{'2}}{2\mu} + V_A(\infty) - V_X(\infty) = \frac{\hbar k^2}{2\mu} - E' + V_A(\infty) - V_X(\infty)
\end{equation}
Now from the equation above we can see that to achieve the maximum energy of the emitted photon, $ \hbar\omega_{max} $ we need to have $ E' = 0 $.
The maximum value of $ E' $ is achieved when $ \hbar\omega = 0 $ therefore:
\begin{equation}
  E_max^{'} = \frac{\hbar k^2}{2\mu} + V_A(\infty) - V_X(\infty)
\end{equation}
Now the \eqref{sumZZ} can be written as:
\begin{equation}
  \sigma = \frac{8}{3}\frac{\pi^2}{c^3k^2}\int_0^{E_{max}^{'}}{dE'\,\omega^3(E')\left[(J+1)M_{J,J+1}^2(k,E')+J\,M_{J,J-1}^2(k,E')\right]}
\end{equation}
where we have the inequality
\begin{equation}\label{sigmalessthan}
  \sigma_{CT} < \frac{8}{3}\frac{\pi^2\omega^3(E')_{max}}{c^3k^2}\int_0^{E_{max}^{'}}{dE'\,\left[(J+1)M_{J,J+1}^2(k,E')+J\,M_{J,J-1}^2(k,E')\right]}
\end{equation}
Now we can replace the expression for $ M_{J+1,J}^2(k,E') $ in \eqref{sigmalessthan} with the expression in \eqref{mjjz} and squaring it,  we get the integral
\begin{equation}
  \int_0^{\infty}{dE'\,JM_{J,J-1}^2(k,E')} = \int_0^{\infty}{dE'\int_{0}^{\infty}{dR\,f_J(kR)D(R)f_{J-1}(k'R}\int_{0}^{\infty}{dR'\,f_J(kR')D(R)f_{J-1}(k'R' }   }
\end{equation}
so the inequality becomes
\begin{equation}
  \int_0^{\infty}{dE'\,JM_{J,J-1}^2(k,E')} < \sum_{E'}{JM_{J,J-1}^2(k,E')} = J\int_{0}^{\infty}{dR\,f_J^2(kR)D^2(R)}
\end{equation}
and we get for the upper bound
\begin{equation}
  \sigma_{CT} < \frac{8}{3}\frac{\pi^2\omega^3(E')_{max}}{c^3k^2}\sum_{J}{(2J+1)\int_{0}^{\infty}{dR\,f_J^2(kR)D^2(R)}}
\end{equation}

\subsection{Optical Potential Approach}

The alternative way to compute the cross section is to use the local optical potential method \eqref{Zygelman88}.  The loss of flux from the entrance channel equals the radiative probability, which is convert to a radiative cross section.
In this method the collision system in the incoming $ A^2\Sigma^{+} $ state experiences imaginary (absorptive) potential $ iA(R)/2 $ added to the entrance-channel BO potential.
So we define a complex potential 
\begin{equation}
\begin{split}
  & V(R) = V_{BO}(R) + V_{OPT}(R) = V_{BO}(R) + i\frac{A(R)}{2} \\[.8em]
  & A(R) = \frac{4}{3c^2}D^(R)[V_A(R) - V_X(R)]
\end{split}
\end{equation}
$ A(R) $ is the Einstein-A coefficient. We shall use a 2D version of the optical theorem to compute the radiative charge transfer in 2D case in the next chapter.
The next step is to solve the usual radial scattering equation, but with $ V(R) $ complex. The resulting S matrix becomes non-unitary, $ |S| < 1 $ and the “missing probability” is the radiative loss.

The cross section for radiative quenching is given by
\begin{equation}
  \sigma = \frac{\pi}{k^2}\sum_J{(2J+1)(1 - e^{-4\eta_j }}
\end{equation}

Local Optical Potential is generally applicable when the following cases are satisfied, which aligns well with our case.
\begin{itemize}
  \item Transition happens over a region small compared to relevant wavelengths
  \item Emission is relatively weak compared to the collision energies
  \item It is accurate enough for the computation of total probability
\end{itemize}

