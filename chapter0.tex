\chapter{Radiative Charge Transfer Processes}
\label{chap:RRP}
\section{Description}
The analysis of the radiative and charge transfer processes is related to the analysis of the general scattering process. So we will briefly touch scattering in the subsequent section.

Typically for the scattering processes at this level, we can distinguish between the radiative and non-radiative processes, i.e. Radiative and Resonant charge transfer.\\

As stated we will focus on a charge transfer processes in low energies, i.e. ultra-cold temperatures. Recently the advances in ultra-cold atom processing technologies have made is feasible to explore the behaviour of ultra-cold atoms and ions in laboratory. Researches are able to explore the interactions between cold ions and neutral atom, including non-radiative and radiative charge transfer as well as radiative association, in which ions and atoms re-combine and re-arrange at cold temperatures. The applications of cold atom chemistry are in astrophysical application, and buffer gas cooling of single ion clocks. The most recent and quite interesting application is the quantum computing. The approach is to  start from identical atoms, cool them to nanokelvin temperatures, and control them with lasers and magnetic fields. However the entanglement is only stable at ultra cold temperatures (milli and nano Kelvins).

The work we do here maps well to the cold atom quantum computing, for example
\begin{itemize}
\item two atom/ion scattering $ \longleftrightarrow $ Collisional gates,
\item Phase shifts $ \longleftrightarrow $  Entanglement phases
\item Dipole transitions $ \longleftrightarrow $ Rydberg gates
\item Optical theorem $ \longleftrightarrow $ Loss and decoherence
\item 2D anyons $ \longleftrightarrow $ Optical lattices 
\end{itemize}

We shall here describe resonant and radiative charge transfer processes as well as the radiative association processes.

We consider \cite{ZygelmanCT} a positively charged Hydrogen $ H_2^+$ ion that approaches a neutral $ H $ atom. As the internuclear distance $ R $ becomes smaller, there exist a non-zero probability that the electron will escape the atom and attach itself to the incoming ion. In this manner a charge transfer happens. \\

Now there are several possible reactions in this case: Radiative Charge Transfer, Resonant Charge Transfer, Radiative Association.

\section{Radiative Charge Transfer}

Radiative charge transfer (abbreviated RCT) is an inelastic, irreversible quantum collision process in which an electron is transferred and a photon is emitted.

In general the process looks like: 
$ A(n+1\prescript{1}{}S) + B \longleftrightarrow A(n\prescript{1}{}S) + B + \hbar\omega $ \\

It is governed by the dipole matrix element $ D(R) = \matrixelement{\psi_g}{r}{\psi_u} $ where $ \ket{\psi_g} $ and $ \ket{\psi_u} $ are the wavefunctions associated with the gerade and ungerade potentials and $ r $ is the location of the electron. 

In our case process is $ H^+ + H \rightarrow H + H^+ + \gamma $. \\

Now the electron jumps from the upper adiabatic curve to the lower one by emitting a photon.  The nuclei continue moving while the electron radiates and drops to the lower curve, after which the electron localizes on one proton.

This is a true inelastic process, controlled by the dipole elements between gerade and ungerade state.

The electric dipole transitions are governed by the selection rules: transitions between electronic states follow parity selection rule. Electric dipole operator is odd (Ungerade). Therefore allowed transitions must change parity: $ g \rightarrow u $. Transitions $ g \rightarrow g $ or $ u \rightarrow u $ are electric dipole forbidden (though allowed by higher multipoles or vibronic coupling).  \\
Orbital interactions: orbitals can only mix (form nonzero overlap/hybridize) if they belong to the same symmetry species, including parity. Thus g orbitals do not mix with u orbitals.

We can consider these charge transfer from the energy point and see that there are, endoergic and exothermic cases.\\
In the endoergic case the binding energy of the electron on the ion is less than that of the atom. In this case,in order to satisfy the energy conservation requirements, an incoming ion needs to supply the additional energy. However if the ion approaches slowly, there is no extra supply of energy. So when an additional energy is not available, charge transfer can only occur if the final electron binding energy is greater than or equal to its binding energy on the atom.\\
If the energy is equal, the reaction is called resonant charge transfer.\\
if not the reaction will processed with the possibility of the emission of a photon.
If radiation is emitted, we call the reaction radiative charge transfer/association and, if not, direct charge transfer.

\section{Radiative Association}

Somewhat related radiative process is Radiative association:

$ A(n+1\prescript{1}{}S) + B \longleftrightarrow A(n\prescript{1}{}S)B + \hbar\omega $ .\\

There is also emission of a photon, but as a result a molecule is formed. In this case a photon removes excess electronic energy and the relative kinetic energy. So in this case, from a  wavefunction point of view, there is a transition from a continuum state to a bound vibrational–rotational state.

This process is dominant at very low collision energies. This is a key mechanism for molecule formation in the interstellar medium.

There are several theoretical approaches and treatments of these classes of problems. \cite{RadQuench1}, \cite{RadQuench2}, \cite{Zygelman88}.

\section{Classical Treatments}

\subsubsection{COB model}
The classical 'over the barrier model' \cite{BransdenMcDowell1992} treats an electron as being inside a potential well. The well's potential is described by the function
\begin{equation}
  V(r) = -\frac{q}{x} - \frac{1}{\lvert R - r \rvert}
\end{equation}
where $ r $ is the distance of the bound electron from the incoming nucleus. $ R $ is the distance between the nuclei, and $ q $ is the charge of the ion.

Potential has it maximum as $ V_{max} = -\left(\sqrt{q} + 1\right)^2/R $. The electron attaches itself to an incoming ion if its binding energy, perturbed by the electron–ion Coulomb energy $ -1/2 - q/R $ becomes greater that $ V_{max} $.

Assuming that the electron's binding energy is $ E_n = -q^2/2n^2 $ where $ n $ is a principal quantum number, energy conservation requires that
\begin{equation}
  -\frac{1}{2} - \frac{q}{R} = E_n - \frac{1}{R}
\end{equation}
Together these relations predict the largest value of $ R $ where the electron is transferred to an incoming ion with an energy $ E_n $ to be:
\begin{equation}\label{COBRT}
  R_n = \frac{2(q-1)}{q^2/n^2-1}
\end{equation}
Assuming the transfer probability of $ 1/2 $ this model predicts the total charge transfer cross section to be:
\begin{equation}
  \sigma_{CT} = \frac{\pi}{2}R_{n^*}^2
\end{equation}
where $ n^* $ is the value of $ n $ in equation \eqref{COBRT} where the equality holds. We shall compare this value with the value obtained by our method. But generally speaking this method fails to give the correct values for the very low energy processes \cite{osti_6533174}.

\subsubsection{Langevin Orbiting}
Since the preceding model fails to give the correct prediction a new, still classical model was suggested by P. Langevin. In this model, the incoming ion introduces the energy shift of the electron's ground state by
\begin{equation}
  \Delta E = -\frac{1}{2}\alpha|E|^2  
 \end{equation}
 where $ \alpha $ is called the polarizability of the atom and $ E = q/R^2 $. The mutually attractive potential has the form :
\begin{equation}
  V_{pol}(R) = -\frac{C_4}{R^4}\,\,\,\,\,C_4 = \frac{\alpha}{2}q^2
\end{equation}
in addition to the centripetal potential
\begin{equation}
  V_L(R) = \frac{L(L+1)}{2\mu R^2}
\end{equation}
where $ L = \mu v b $ is an angular momentum and $ b $ is an impact parameter. For a given $ b $, the effective potential, the sum of $ V_{pol} $ and $ V_L $ possesses a local maximum $ V_{max}(b) $ at critical value $ R_c $.
At the end this model predicts the total charge transfer cross section to be:
\begin{equation}
  \sigma_{CT} = 2\pi \int_0^{b_c}{db\, b} = \pi b_c^2 = \frac{\pi}{v}\sqrt{\frac{8 C_4}{\mu}} = \frac{2\pi q}{v}\sqrt{\frac{\alpha}{\mu}}
\end{equation}

\subsubsection{Landau–Zener–Rosen Theories}
This is a more advanced method which takes into account quantum nature of the particles. We shall skip the explanation, just to note that this model predicts the cross section to be:
\begin{equation}
\begin{split}
  & \sigma_{CT} = 2\pi \int_0^{R_c}{b P(b) db} \\[.8em]
  & P(b) = \frac{1}{2}sech^2\left(\frac{\pi \Delta E(R_c)}{2\lambda v(b)}\right)
\end{split}
\end{equation}

subsection{Quantum Mechanical Treatment, MOCC Approach}

The Molecular Orbital Approach (MOCC) assumes that the total system amplitude is expressed as a sum of electronic
eigenstates, whose expansion coefficients describe the motion of the nuclear coordinate R. There are two versions of
this theory. In the first, the nuclear motion is treated classically, but the electronic degrees of freedom are treated quantum mechanically. This approach, the method of perturbed stationary states (PSS), was first advocated by Massey and Smith in 1932.
The quantum version of this theory treats both the nuclear motion and the electronic degrees of freedom quantum mechanically. It was developed by David Bates and coworkers in the 1950s. It was not much used since it required a sizable computing power. So only in the 80s the fully quantum MOCC computation becomes available. \cite{ZygelmanCT}.

In the fully quantum expressions of the PSS equations we get for the wavefunction

\begin{equation}
  \Psi(\mathbf{r},\mathbf{R}) = \sum_n^N{\phi_n(\mathbf{r},\mathbf{R})F_n(R)}
\end{equation}
where $ F_n(R) $ are quantal amplitudes that predict the prob- ability for the system to occupy the n-th electronic states
at R. Index $ N $ basically includes all the states that are energetically open.
The states $ \phi_n $ form a complete set
\begin{equation}
  \sum_n{\phi_n^{\dag}(\mathbf{r},\mathbf{R})F_n(R)\phi_n(\mathbf{r'},\mathbf{R})F_n(R)} = \delta(\mathbf{r} - \mathbf{r'})
\end{equation}

While the problem has not been solved analytically, we will show that it can be solved numerically, to a desired precision. The radiative processes are driven by the interaction of the collision system with the radiation field. The direct change transfer is due to the transition between atomic (molecular) states due to the nuclear motion. Because the collision energy considered is low, typically only molecular states included are those which correspond to the initial $ A \prescript{1}{}\Sigma^+ $  and final $ X \prescript{1}{}\Sigma^+ $ channels.

\subsection*{Length gauge}

This is an interesting approach, which could potentially be used in calculation. In this thesis we would not be using it.

The length gauge is a gauge transformation that replaces the vector potential for the field by the scalar potential for the quasi-static electric field \cite{LengthGauge3}.  In this gauge we take the Hamiltonian as: $ H = \mathbf{p}^2/2m + V(\mathbf{r})  + e\mathbf{E}\mathbf{r} $. The length gauge is convenient since both the Coulomb and the external fields are represented by the scalar potentials, which are additive. In the presence of the radiation field, the length gauge is obtained by the gauge transformation of the vector potential $ \mathbf{A} $, such that $ \mathbf{A} \rightarrow \mathbf{A} + \nabla \chi $ where $ \chi = - \mathbf{r} \cdot \mathbf{A} $.

In the length gauge, the interaction Hamiltonian is:
\begin{equation}
\begin{split}
& H_{int} = -\sum_j{ \mathbf{r}\cdot\mathbf{E} } \\[.8em]
& \mathbf{E} = i\,\sum_{k\alpha}{\left(\frac{2\pi c k}{V}\right)^{1/2}\hat{\epsilon}_{k\alpha}\left(a_{k\alpha} - a^{\dagger}_{k\alpha}\right)}
\end{split}
\end{equation}
where $ a_{k\alpha} $ and $ a^{\dagger}_{k\alpha} $ are destruction and creation operators for the photon of momentum $ \hbar k $ and polarization $ \alpha $ respectively.

\section{Radiative Association}

\subsection{Motivation}

Fairly recently there has been advances in trapping, cooling, and  manipulation of ultra-cold atom \cite{C1CP21205B}. The development of hybrid, ion-atom traps has allowed researchers to explore the competing pathways, including non-radiative and radiative charge transfer as well as radiative association, in which ions and atoms re-combine and re-arrange at cold temperatures. 

Some of the possible applications are in astrophysics \cite{Zygelman1}, single ion clocks and recently the quantum computing.  Ultracold atoms, cooled to nanokelvin temperatures near absolute zero $ \approx \mu K $  using laser cooling and evaporative techniques, serve as a premier platform for quantum simulation and computing. By trapping atoms in optical lattices or tweezers, they act as stable qubits for quantum information processing and simulate complex, many-body systems. 

There has been some recent \cite{C1CP21205B} laboratory study that used used a hybrid trap to investigate the interactions among cold $ {^174}Yb^+ $ ions with $ {}^{40}Ca $ atoms. That study reported large, $ 2\times 10^{-10} cm^3s^{-1} $ chemical reaction rate coefficient and, based on their calculations, argued that radiative charge transfer is the dominant process that contributes to this rate. This is several orders of magnitude larger than similar reactions for the lighter species. The paper by\cite{ZygelmanHunt2012} repeated the calculation using the optical potential method \cite{Zygelman88} for the total of two reactions.
\begin{equation}\label{Eq1}
\begin{split}
  & Yb^+ + Ca \rightarrow Yb + Ca^+ + \hbar\omega \\[.8em]
  & Yb^+ + Ca \rightarrow YbCa^+ + \hbar\omega 
\end{split}
\end{equation}

We will show that our results  are several orders of magnitude smaller than that given in \cite{C1CP21205B}. Based on these results there are some doubts on the conclusion, given in that paper, concerning the role of radiative relaxation.  Following the approach in the \cite{ZL} we apply a semi-classical approach to the system of $ YbCa $ molecules. This approach, at utra low energies (temperatures) hasn't really been tested before. 
In addition, we used molecular data that was gleaned from the illustrations given in \cite{C1CP21205B}. At these energies small
details in the potential surfaces and dipole moments can be important. For these reasons, we have redone the calculations here using the original data given in \cite{C1CP21205B}. We compare the predictions of the local-optical potential with the results obtained using the Fermi-Golden-Rule prescription \cite{Zygelman88}.
The result obtained by our calculation in the subsequent chapter are are several orders of magnitude smaller than that given in \cite{C1CP21205B}. 

The local optical potential method has it roots in the semi-classical theory of radiative association first developed by Kramers and Ter-Haar \cite{KramersTerHaar1946}.  In the subsequent discussion, and it that theory we assume that a photon is emitted with energy equal to the energy difference between two Born-Oppenheimer potential surfaces, at the distance where the transition occurs. The efficiency of a transition is determined by an Einstein-A coefficient at that internuclear distance.
This leads to the Local Optical Potential Method, and following the semi-classical approach, the total rate is estimated as a classical integral over all localized transitions.\\

In our discussion below we briefly summarize the various theoretical approaches and use them to calculate the rates for processes.\\
We a provide a rigorous upper bound for the sum of rates given in \eqref{Eq1}.

\subsection{Theory}
\label{chap:RxRP}

Here we follow the discussion in \cite{ZL}, and apply the theory to the 3D YbCa molecule.
The cross section of the spontaneous radiative association is given by \cite{Zygelman1} .  This can also be derived using the Fermi's Golden Rule (FGR)(which is actually published by Dirac 20 years before Fermi)\cite{FermiGR}:
Considering the radiative association process:
\begin{equation}
  Yb^+ + Ca \rightarrow YbCa^+ + \hbar\omega
\end{equation}
where $\hbar\omega $ is the energy of the emitted photon. The cross section for the radiative association process is given by:
\begin{equation}\label{eq3}
\begin{split}
  &  \sigma_{RA} = \sum_J{\sum_n{\frac{8}{3}\frac{\pi^2\omega_{nJ}^2}{c^3k^2} \left[(J+1)M_{J+1,J}^2(k,k')+J\,M_{J-1,J}^2(k,k')\right]}} \\[.8em]
  & M_{J,J'}(k,n) = \int_{0}^{\infty}{dR\,f_J(kR)D(R)\phi_{J'n}(R) }
\end{split}
\end{equation}
where $ D(R) $ is the transition dipole moment  between $ X^2\Sigma^{+} $  and $ A^2\Sigma^{+} $ states of the $ YbCa^+  $ molecular ion. $ \phi_{J'n}(R) $ is a rho-vibrational eigenstate of the $ X^2\Sigma^{+} $ ground state, with energy eigenvalue $ \epsilon_{nJ} $. The $ J $ and $ n $ represent the angular and vibrational momentum quantum numbers respectively. $ f_J(kR) $ is s the wavefunction that satisfies the
radial Schrodinger equation
\begin{equation}\label{eq4}
  f_{J}''(kR) - \frac{J(J+1)}{R^2}f_J(kR) + 2\mu V_{A}(R))f_{J}(kR) + k^2 f_{J}(kR) = 0
\end{equation}
where $ V_{A}(R) $ is the Born-Openheimer energy of the excited $ A^2\Sigma^{+} $ state, $ \mu $ is the reduced mass of the collision system and $ k $ s the wavenumber for the incident collision partners in that channel.
The asymptotic solution of the $ f_J(kR) $ has the form:
\begin{equation}
f_{J}(kR) \longleftarrow \sqrt{\frac{2\mu}{\pi k}}\sin(kR - \frac{J\pi}{2} + \delta_J)
\end{equation}
where $ \delta_J $ is a phaseshift, as $ R \rightarrow \infty $.
The energy of the emitted photon is given by
\begin{equation}
\hbar\omega = \frac{\hbar k^2}{2\mu} + V_A(\infty) - \epsilon_{nJ} - V_X(\infty) 
\end{equation}

This process is governed by the selection rules, 
\begin{itemize}
\item Rotational: $ \Delta J = \pm 1 $
\item Electric Dipole: $ \mel{f}{D}{i} $ 
\item Parity: $ g \leftrightarrow u $
\item Electronic angular momentum: $ \Delta \Lambda = 0, \pm 1 $
  \begin{itemize}
    \item where $ \Lambda = 0,1,2 $ corresponds to $ \Sigma, \Pi, \Delta $ states
  \end{itemize}
\end{itemize}

\begin{figure}[H]
  \includegraphics[width=1.0\textwidth]{RadAssocPotentials-4}
  \caption{Illustration of the BO molecular potential curves (solid thick lines) participating in the radiative association process. In the $ A^2\Sigma^+ $ entrance channel the wave function is shown by the light undulating line. The oscillations are due to the strong polarization force in the entrance channel leading to a potential minimum at $ R \approx 14a_0 $.  Association is precipitated by the emission of a photon of energy $ \omega $ near the classical turning point. The final bound rho-vibrational state, in the $ X^2\Sigma^+ $ channel, is shown by the thin line }
  \label{RadAssoc4}
\end{figure}

\section{Radiative Charge Transfer theory}
\label{chap:RxCP}

Here we consider  radiative charge transfer process
\begin{equation}
  Yb^+ + Ca = Yb + Ca^+ \hbar\omega
\end{equation}
where $ \hbar\omega $ is the energy of the emitted photon given by the expression \eqref{hbaromega2}.
The cross section for the radiative charge transfer process is given by
\begin{equation}\label{sumZZ}
\begin{split}
& sigma_{CT} = \int_0^{\omega_max}{d\omega\,\frac{d\sigma}{d\omega}},\\[.8em]
& \frac{d\sigma}{d\omega} = \sum_J{\frac{8}{3}\frac{\pi^2\omega_{nJ}^3}{c^3k^2}\left[(J+1)M_{J+1,J}^2(k,k')+J\,M_{J-1,J}^2(k,k')\right] }
\end{split}
\end{equation}
where
\begin{equation}\label{mjjz}
  M_{J,J'}(k,n) = \int_{0}^{\infty}{dR\,f_J(kR)D(R)f_J'(k'R }
\end{equation}
Here $ f_J(kR) $ is a solution to \eqref{diffR} and $ f_J^{'}(k^{'}R) $ obeys the corresponding equation for the $ X^2\Sigma^+ $ exit channel with wavenumber and partial wave $ k' $, $ J'$ respectively.
The radial wavefunctions are normalized as in \eqref{fjRZ} and
\begin{equation}\label{hbaromega2}
\begin{split}
  & \hbar\omega = \frac{\hbar k^2}{2\mu} - \frac{\hbar k^{'2}}{2\mu} + \Delta E \\[.8em]
  & \Delta E = V_A(\infty) - V_X(\infty)
\end{split}
\end{equation}
From the \eqref{hbaromega2} the maximum angular frequency $ \omega_max $ is
\begin{equation}
\hbar\omega_{max} = \frac{\hbar k^2}{2\mu} +  \Delta E
\end{equation}
We avoid evaluating the sum \eqref{sumZZ}. \\

Instead we will compute an upper bound on the charge transfer cross section.
From the equation above we get for the frequency of the emitted photon, during an RCT transition, to be
\begin{equation}
\hbar\omega = \frac{\hbar k^2}{2\mu} - \frac{\hbar k^{'2}}{2\mu} + V_A(\infty) - V_X(\infty) = \frac{\hbar k^2}{2\mu} - E' + V_A(\infty) - V_X(\infty)
\end{equation}
Now from the equation above we can see that to achieve the maximum energy of the emitted photon, $ \hbar\omega_{max} $ we need to have $ E' = 0 $.
The maximum value of $ E' $ is achieved when $ \hbar\omega = 0 $ therefore:
\begin{equation}
  E_max^{'} = \frac{\hbar k^2}{2\mu} + V_A(\infty) - V_X(\infty)
\end{equation}
Now the \eqref{sumZZ} can be written as:
\begin{equation}
  \sigma = \frac{8}{3}\frac{\pi^2}{c^3k^2}\int_0^{E_{max}^{'}}{dE'\,\omega^3(E')\left[(J+1)M_{J,J+1}^2(k,E')+J\,M_{J,J-1}^2(k,E')\right]}
\end{equation}
where we have the inequality
\begin{equation}\label{sigmalessthan}
  \sigma_{CT} < \frac{8}{3}\frac{\pi^2\omega^3(E')_{max}}{c^3k^2}\int_0^{E_{max}^{'}}{dE'\,\left[(J+1)M_{J,J+1}^2(k,E')+J\,M_{J,J-1}^2(k,E')\right]}
\end{equation}
Now we can replace the expression for $ M_{J+1,J}^2(k,E') $ in \eqref{sigmalessthan} with the expression in \eqref{mjjz} and squaring it,  we get the integral
\begin{equation}
  \int_0^{\infty}{dE'\,JM_{J,J-1}^2(k,E')} = \int_0^{\infty}{dE'\int_{0}^{\infty}{dR\,f_J(kR)D(R)f_{J-1}(k'R}\int_{0}^{\infty}{dR'\,f_J(kR')D(R)f_{J-1}(k'R' }   }
\end{equation}
so the inequality becomes
\begin{equation}
  \int_0^{\infty}{dE'\,JM_{J,J-1}^2(k,E')} < \sum_{E'}{JM_{J,J-1}^2(k,E')} = J\int_{0}^{\infty}{dR\,f_J^2(kR)D^2(R)}
\end{equation}
and we get for the upper bound
\begin{equation}\label{sigmaUpper}
  \sigma_{CT} < \tilde{\sigma}_{CT}\frac{8}{3}\frac{\pi^2\omega^3(E')_{max}}{c^3k^2}\sum_{J}{(2J+1)\int_{0}^{\infty}{dR\,f_J^2(kR)D^2(R)}}
\end{equation}

\subsection{Optical Potential Approach}

The alternative way to compute the cross section is to use the local optical potential method \eqref{Zygelman88}.  The loss of flux from the entrance channel equals the radiative probability, which is convert to a radiative cross section.
In this method the collision system in the incoming $ A^2\Sigma^{+} $ state experiences imaginary (absorptive) potential $ iA(R)/2 $ added to the entrance-channel BO potential.
So we define a complex potential 
\begin{equation}
\begin{split}
  & V(R) = V_{BO}(R) + V_{OPT}(R) = V_{BO}(R) + i\frac{A(R)}{2} \\[.8em]
  & A(R) = \frac{4}{3c^2}D^(R)[V_A(R) - V_X(R)]
\end{split}
\end{equation}
$ A(R) $ is the Einstein-A coefficient. We shall use a 2D version of the optical theorem to compute the radiative charge transfer in 2D case in the next chapter.
The next step is to solve the usual radial scattering equation, but with $ V(R) $ complex. The resulting S matrix becomes non-unitary, $ |S| < 1 $ and the “missing probability” is the radiative loss.

The cross section for radiative quenching is given by
\begin{equation}\label{quenchR}
  \sigma = \frac{\pi}{k^2}\sum_J{(2J+1)(1 - e^{-4\eta_j)}}
\end{equation}

where $ \eta_j $ is the imaginary part of the J'th partial wave phase shift $\delta_j $ for the radial wave $ f_J(kR) $ that satisfies
\begin{equation}\label{PartW3D}
  f_J^{''}(kR) - \frac{J(J+1)}{R^2}f_J(kR) + 2\mu (V_A(R) + V_{opt}(R))f_J(kR) + k^2f_J(kR) = 0
\end{equation}

Local Optical Potential is generally applicable when the following cases are satisfied, which aligns well with our case.
\begin{itemize}
  \item Transition happens over a region small compared to relevant wavelengths
  \item Emission is relatively weak compared to the collision energies
  \item It is accurate enough for the computation of total probability
\end{itemize}

\subsection {Ultra Cold Limit}

In the limit of ultra cold temperatures, only $ s $ wave participates in the process, so the total cross section takes arguably simpler form.
\begin{equation}\label{coldInt}
  \sigma = \sum_{n}{\frac{16\mu \pi \omega_n^3}{3 c^3 k}\left|\int_0^{R_0}{dR\,\phi(R)D(R)\phi_{J=1}^{n}(R)} \right|^2 }
\end{equation}

where $ \phi(R) $ is the s-wave solution to \eqref{eq4}. 
The boundary condition for $ \phi(R) $
\begin{equation}
\frac{d\phi(R)}{dR}|_{R_0} = 1
\end{equation}
is valid for a sufficiently large radius $ R $ and $ \phi_{J=1}^{n}(R) $ are $ J = 0 $ rho-vibrational states of the $ X^2\Sigma^+ $ potential.
Because there is only s-wave, the integral \eqref{coldInt} does not depend on the incoming wave number $ k $. Therefore the equation \eqref{coldInt} predicts that the association cross section, in the ultra cold regime scales as the inverse of the incoming velocity, i.e. the rate tends to constant.

We shall show in the subsequent chapter the method used to compute the phase shift.But basically the idea is to match the numerical solution for the $ f_J(kR) $ with the asymptotic form, for the large values of $ R $.

\begin{figure}[H]
  \includegraphics[width=1.0\textwidth]{EnsteinA3D}
  \caption{R-dependent Einstein A coefficient as a function of internuclear distance}
\end{figure}

\section{Results}

Figure \ref{RadAssoc4} illustrates the mechanism for radiative association for the $ Yb^+ $ ion and $ Ca $ atom that approach in the $ A^+\Sigma^2 $  electronic BO state. The BO energies where taken from the data of the ab-initio calculations reported in i\ref{Ref4}. At large internuclear distances this potential has the form
\begin{equation}
  V_A(R) \rightarrow \frac{C_4^A}{R^4}\,\,\,\,\,C_4^A = 78.5
\end{equation}

In the case of Radiative association the incident $ A^2\Sigma^+ $ channel can relax via an emission of a photon. In that case the final state is a bound rho-vibrational level of the $ X^2\Sigma^+ $  channel.  In radiative charge transfer the collision partners can exit in that channel, as a re-arranged $ Yb - Ca^+ $  pair. In the exit channel
\begin{equation}
  V_X(R) \rightarrow \frac{C_4^X}{R^4} + \Delta E\,\,\,\,\, C_4^X = 71.5\,\,\,\,\, \Delta E = -0.0052
\end{equation}
as $ R \rightarrow \infty $ \\
In calculating the radiative association cross sections given by \eqref{eq3} we need to itemize all bound
states  supported by the $ X^2\Sigma^+ $ channel. The total number of bound states can be approximated using the JWKB expression
\begin{equation}
  n = \text{Floor}\left[\int_{R_c}^{\infty}{dR\,\sqrt{-2\mu (V_X(R) - \Delta E)} - 1/2}\right] \approx 54,878
\end{equation}
where $ R_c $ is a classical turning point Floor[x] is the integer lower bound of $ x $.\\
Here we have a large reduced mass $ \mu $, i.e. the participants mass is large which in turns produces a very large number of the bound states. For a lighter participants the number of rho-vibrational states is typically in the order of a few hundred \cite{Zygelman1}.\\
Here we have another factor to consider. In the ultra-cold temperatures the participant have lower energies so the centrifugal repulsion in the entrance channel limits the number of partial waves that participate. And since the selection rules $ J \pm 1 $ still applies, it constrains the  possible rho-vibrational levels. For example, at a collision energy corresponding to a temperature of $ 1 mK $, only levels with $ J $ up to the value $ \approx 15 $ contribute to the association rate.\\

The following graph \ref{compComp2} shows the results of the calculation.

\begin{figure}[H]
  \includegraphics[width=1.0\textwidth]{ResultsYbCa}
  \caption{Plot of various cross sections as a function of incoming partial wave $ J $. Circles represent data for the radiative association cross sections, X’s represent and upper bound for the total (association + radiative charge transfer), and squares represent the data for the total radiative relaxation obtained using the optical potential method.}
  \label{compFig3}
\end{figure}

We have made the calculations two ways. First starting from the equation \eqref{eq4} we computed the cross section using equation \eqref{sumZZ}. In the figure \ref{compFig3} symbol $ X $ represents the upper limit for total radiative relaxation, which is obtained by adding the association cross sections \eqref{sumZZ} with those given by expression \eqref{sigmaUpper}.\\
The square icons represent the cross sections predicted by expression \eqref{quenchR},i.e. by the optical potential method.\\
So from figure \ref{compFig3} for values $ J < 10 $ we see that an optical potential method provides a very good approximation to the total cross section.\\
However this approximation is good only for values $ J < J_{max} $ \cite{ZygelmanHunt2012}.
\begin{equation}
  J_{max} = \sqrt[4]{8\mu k^3 C_4^A} = \sqrt[4]{24 \mu^2 k_B T C_4^A} \approx 12
\end{equation}
The collision energy of the collision system is given as $ 3/2 k_B\, T $ where $ k_B $ is the Boltzman constant and $ T $ is the temperature in Kelvin \\
Value $ J_{max} $ is the critical angular momentum for which the collision system has sufficient collision energy to overcome the centrifugal potential barrier \cite{ZygelmanHunt2012}. For larger $ J $ tunneling resonances can access the inner region where the transition dipole moment is non-negligible and induce a radiative transition.

In Table I in the \ref{ZL} we can see tabulated the various cross sections at several representative collision temperatures. The second column contains the association cross section obtained using the FGR method described above. For the radiative charge transfer cross sections, itemized in the third column, we used \eqref{sigmaUpper}. Thus the upper bound for the total radiative relaxation cross sections are given in column 4. The last column gives the results obtained using the local optical potential method. The table shows that, over the temperature range considered, the local optical potential method predicts cross sections that are less than the upper bound itemized in column 4. Secondly, the differences between the predictions of the two theories are small. The optical potential cross section differs by less than 4\%  from the upper limit values over the entire temperature range, including the ultra-cold region.\\
One can also note that the optical potential method predicts cross sections that are larger than the radiative association cross sections. At the same time the optical potential method provides a reliable upper bound for the total (RR) cross section.\\

In Figure  we plot the total radiative relaxation cross section, obtained using the optical potential method, for the gas temperature range $ 1mK < T < 1K $.

\begin{figure}[H]
  \includegraphics[width=1.0\textwidth]{CrossSection3D}
  \caption{Total radiative relaxation cross section as a function of collision energy expressed as $ E = 3/2\,k_B T $, where T is the temperature in Kelvin}
  \label{Fig4}
\end{figure}

Although the optical potential method provides a good approximation for the total radiative relaxation rate, calculation of the photon emission spectrum requires the use of the FGR method. use of the FGR method. In \ref{Fig5} we illustrate the association cross sections $ \sigma_{n\,J} $ at $ T = 1mK $ for the individual rho-vibrational levels as function of the frequency of the emitted photon. The structure of the emission pattern is the result of the oscillation in the incoming wave as shown in \ref{RadAssoc4}

\begin{figure}[H]
  \includegraphics[width=1.0\textwidth]{Spectrum3D}
  \caption{mission spectrum for the radiative association process at a gas temperature of 1mK.}
  \label{Fig5}
\end{figure}

In the limit $ T \rightarrow 0 $ we define a complex scattering length for the s--wave solution to \eqref{PartW3D}
\begin{equation}
  a \equiv - \frac{1}{k}\tan \delta(k) = 10785 - i0.008842;\,\,\,\,\text{as}\,\,\,k \rightarrow 0
\end{equation}
Therefore, the total RR cross section, according to the optical potential method, has the limiting value
\begin{equation}
  \sigma = \frac{4\pi}{k} 8.842 \times 10^{-3}  
\end{equation}
Defining the rate coefficient 
\begin{equation}
  k_{RR} \equiv \langle \nu \sigma \rangle
\end{equation}
we use the results from \cite{ZL} to show the values of $ k_{RR} \approx 1.15 \times 10^{-14} $.\\
By looking at \ref{Fig4} we observe that $ k_{RR} $ varies slow, from $ 1.6 \times 10^{-15}cm^3 s^{-1} $ at 1mK to $ 2 \times 10^{-15} cm^3 s^{-1} $ at 1K.

We can summarize the current chapter as following. Using both quantum and semi-classical methodologies, we calculated the rates for the radiative association and charge transfer in cold collisions of $ Yb^+ $ with $ Ca $. We have shown that the optical potential method provides very good results in predictions for the total radiative relaxation rates, in the ultra-cold s-wave scattering regime
In addition we have computed and shown the cross section $ \sigma $ as function of temperature $ K $ and frequency of the emitted photons .

\subsection{Isotope Dependency}

Different Yb isotopes $ ({}^{170}Yb, {}^{174}Yb, {}^{176}Yb) $ differ only in nuclear mass, not in electronic structure. Ergo Isotope dependence enters through nuclear motion. Following \cite{ZygEtAll4} we observe that radiative processes $ Yb^+ + Ca \rightarrow Yb + Ca^+ + \hbar\omega $ and $ Yb^+ + Ca \rightarrow YbCa^+ + \hbar\omega $ are both controlled by the overlap integral
\begin{equation}
  \int_0^{R_0}{dR\,\phi(R)D(R)\phi_{J=1}^{n}(R)}
\end{equation}
Because isotope mass shifts the bound-state energies, it leads to order-of-magnitude differences in rate constants between isotopes.\\
At ultralow temperatures ($ \mu K $K and below), only S-wave is present rates obeys Wigner threshold law. In that case, if a Yb isotope supports a bound state just below threshold, the scattering length becomes very large, the radiative association rate is strongly enhance. The enhancement becomes isotope specific. This explains the pronounced isotope separation seen in calculated rate curves as shown in \cite{ZygEtAll4}.
So Radiative association shows strong isotope dependence while Radiative charge transfer isotope dependence is smaller.

In the following sections we shall start focusing on the 2D systems and describe Resonant Charge Transfer. Something we will calculate in the following sections.


