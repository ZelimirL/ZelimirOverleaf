\chapter{ Hydrogen Ion in 2 Dimension Equation, Derivation }
\label{AppendixC}

Here we solve the equations \eqref{eqLG3}
\begin{equation}\label{eqPartial2DG}
\left(-\frac{1}{2}\nabla^2-\frac{1}{r_a}-\frac{1}{r_b}\right)\psi = E\,\psi
\end{equation}

As illustrated by the figure \ref{h2ion2d}, we express equation \eqref{eqPartial2D} in the elliptical coordinates, and by setting the $ x $ axis to be perpendicular to the internuclear axis, we have the nuclei at: $ y = \pm \frac{R}{2}  $, R being the distance between nuclei. So in  2D elliptic coordinates, $ \lambda $, $ \mu $ we have 
\begin{equation}\label{variables1G}
\begin{split}
& \lambda = \left(r_a + r_b\right)/R;\,\,\,\,\,\,\,\,\,\,\,\,\,\,\,\,\,\,\mu =  \left(r_a - r_b\right)/R  \\[1em]
& \lambda = \left(r_a + r_b\right)/R;\,\,\,\,\,\,\,\,\,\,\,\,\,\,\,\,\,\,\mu =  \left(r_a - r_b\right)/R  \\[1em]
& \text{where } \lambda \in \left[1,\infty\right]\,,\,\,\,\,\,\,\,\,\,\mu \in \left[ -1, 1 \right]\,\,\,\,\,\,\,\,\,\text{ and } \\[.8em] 
& r_a = \frac{R}{2}\left(\lambda + \mu \right)\,\,\,\,\,\,\,\,\,\,\,\,\,\,\,\,\,\,\,\,\,\, r_b = \frac{R}{2}\left(\lambda - \mu \right)
\end{split}
\end{equation}

Plug the variables $ \lambda $ and $ \mu $ into the equation \eqref{eqPartial2DG} we get by using 3D elliptic cylindrical coordinates \cite{MorseFeshbach} .
\begin{equation}
\begin{split}
& \nabla^2 = \frac{4}{ R^2 (\lambda^2-\mu^2) }\left[ \sqrt{\lambda^2-1}\frac{\partial}{\partial \lambda}\left(\sqrt{\lambda^2-1}\frac{\partial \psi}{\partial \lambda} \right) + 
\sqrt{1-\mu^2}\frac{\partial}{\partial \mu}\left(\sqrt{1 - \mu^2}\frac{\partial \psi }{\partial \mu }\right) \right] = \\
& \frac{4}{ R^2 (\lambda^2-\mu^2) }\left[(\lambda^2-1)\frac{\partial^2 \psi}{\partial \lambda^2} + \lambda\frac{\partial \psi}{\partial \lambda} + (
1 - \mu^2)\frac{\partial^2 \psi}{\partial \mu^2} - \mu\frac{\partial \psi}{\partial \mu} \right]
\end{split}
\end{equation}

So the equation transforms to (with $ \psi = \psi(\lambda, \mu) $:
\begin{equation}
-\frac{1}{2}\frac{4}{ R^2 (\lambda^2-\mu^2) }\left[(\lambda^2-1)\frac{\partial^2 \psi}{\partial \lambda^2} + \lambda\frac{\partial \psi}{\partial \lambda} + 
(1 - \mu^2)\frac{\partial^2 \psi}{\partial \mu^2} - \mu\frac{\partial \psi}{\partial \mu} \right] - \frac{2}{R(\lambda+\mu)}\psi - \frac{2}{R(\lambda-\mu)}\psi = E \psi
\end{equation}
or
\begin{equation}\label{SchrFull-1}
-\frac{1}{2}\frac{4}{ R^2 (\lambda^2-\mu^2) }\left[(\lambda^2-1)\frac{\partial^2 \psi}{\partial \lambda^2} + \lambda\frac{\partial \psi}{\partial \lambda} + 
(1 - \mu^2)\frac{\partial^2 \psi}{\partial \mu^2} - \mu\frac{\partial \psi}{\partial \mu} \right] - \frac{4}{R}\frac{\lambda}{\lambda^2-\mu^2}\psi = E \psi
\end{equation}

We assume that the total electronic wavefunction can be written as the product of two functions:
\begin{equation}\label{variables2C}
\psi(\lambda,\mu) = F(\mu)L(\lambda)
\end{equation}
we obtain from \eqref{SchrFull-1}
\begin{equation}
\begin{split}
& \frac{2}{ R^2 (\lambda^2-\mu^2) }\left[(\lambda^2-1)F\frac{\partial^2 L}{\partial \lambda^2} + \lambda F\frac{\partial L}{\partial \lambda} + 
(1 - \mu^2)L\frac{\partial^2 F}{\partial \mu^2} - \mu L\frac{\partial F}{\partial \mu} \right] + \frac{4}{R}\frac{\lambda}{\lambda^2-\mu^2} F L + E F L = 0 \Rightarrow \\[.8em]
& (\lambda^2-1)\frac{1}{L}\frac{\partial^2 L}{\partial \lambda^2} + \frac{\lambda}{L}\frac{\partial L}{\partial \lambda} + 
(1 - \mu^2)\frac{1}{F}\frac{\partial^2 F}{\partial \mu^2} - \frac{\mu}{F} \frac{\partial M}{\partial \mu} + 2R\lambda + \frac{R^2}{2} E (\lambda^2 - \mu^2) = 0 
\end{split}
\end{equation}
Setting $ p^2 = \frac{R^2}{2}E $ and rearraging: \\[1.em]
\begin{equation}
\begin{split}
& (\lambda^2-1)\frac{1}{L}\frac{\partial^2 L}{\partial \lambda^2} + \frac{\lambda}{L}\frac{\partial L}{\partial \lambda} + 2R\lambda + \frac{R^2}{2} E \lambda^2 + \\[.8em]
& (1 - \mu^2)\frac{1}{F}\frac{\partial^2 F}{\partial \mu^2} - \frac{\mu}{F} \frac{\partial M}{\partial \mu} - \frac{R^2}{2} E \mu^2 = 0
\end{split}
\end{equation}
We can conclude that both equations for $ \lambda $ and $ \mu $ must be equal to the separation constant $ A $. So it follows that
\begin{equation}\label{eqLG}
\begin{split}
& (\lambda^2-1)\frac{\partial^2 L}{\partial \lambda^2} + \lambda\frac{\partial L}{\partial \lambda} + 2R\lambda L - p^2 \lambda^2 L = AL \\[.8em]
& (1 - \mu^2)\frac{\partial^2 F}{\partial \mu^2} - \mu\frac{\partial F}{\partial \mu} - p^2\mu^2 F = -AF
\end{split}
\end{equation}
where $ A $ is the separation constant.

Now solve each equation.

\section{$ \mu $ Equation}
Using the substition $ \mu = \cos x $, $ d\mu = -\sin x dx $,  we get the other form of the equation:
\begin{equation}
\frac{dF}{d\mu} = -\frac{1}{\sin x}\frac{d M}{d x}
\end{equation}
\begin{equation}
\begin{split}
& \frac{d^2F}{d\mu^2} = \frac{d}{d x}\left(-\frac{1}{\sin x}\frac{d M}{d x}\right)\frac{d x}{d \mu} = \frac{d}{d x}\left(-\frac{1}{\sin x}\frac{d M}{d x}\right)\left(-\frac{1}{\sin x}\right) = \left[\frac{\cos x}{\sin^2 x}\frac{d M}{d x} - \frac{1}{\sin x}\frac{d^2 M}{d x^2}\right]\frac{-1}{\sin x} \Rightarrow \\[.8em]
& \frac{d^2 F}{d\mu^2} = \frac{1}{\sin^2 x}\frac{d^2 M}{d x^2} - \frac{\cos x}{\sin^3 x}\frac{d M}{d x}
\end{split}
\end{equation}
Plug in equation for $ F(\mu) $.
\begin{equation}\label{Feq}
\begin{split}
& \frac{d^2 M}{d x^2} - \frac{\cos x}{\sin x}\frac{d M}{d x} + \frac{\cos x}{\sin x}\frac{d M}{d x} + \left(-A + p^2\cos^2 x\right) M = 0 \Rightarrow \\[.8em]
& \frac{d^2 M}{d x^2} + \left(-A + p^2\cos^2x\right)M = 0
\end{split}
\end{equation}
Using $ \cos(2x) = \cos^2x - \sin^2x = 2\cos^2x - 1 $ we get:
\begin{equation}\label{Feq2-C}
\frac{d^2 M}{d x^2} + \left[-A + \frac{p^2}{2} + \frac{p^2}{2}\cos(2x) \right]M = 0 
\end{equation}

\section{$ \lambda $ Equation}

 So assume the solution as the sum of Laguerre polynomials and Algebra:
\begin{equation}
\begin{split}
& L(x) =  e^{-px}\sum_{k=0}^{\infty}{c_k\,L_k(x)}\,\,\text{ with: ( the prime is a derivative) } \\[.8em]
& x\,L_k^{'}(x) = k\,L_k(x)  - k\,L_{k-1}(x) \\[.8em]
& x\,L_k^{''}(x) = (x-1)L_k^{'}(x) - k\,L_k(x) = xL_k^{'}(x)-L_k^{'}(x) - n=k\,L_k(x) = \\[.8em] 
& = k\,L_k(x)  - k\,L_{k-1}(x) - L_k^{'}(x) - k\,L_k(x) = - L_k^{'}(x) - k\,L_{k-1}(x)\\[.8em]
\end{split}
\end{equation}\\*
First, we shift the domain $ \lambda = x + 1 $. The equation is:
\begin{equation}
\begin{split}
x(x +2)\frac{d^2\,L}{d\,x^2} + (x+1)\frac{d\,L}{d\,x} + \left[ -p^2x^2  -2p^2x + 2Rx - p^2 + 2R - A \right] L = 0;
\end{split}
\end{equation}\\*
And the derivatives are:
\begin{equation}
\begin{split}
& L^{'} =  -p\,e^{-px}\sum_{k=0}^{\infty}{c_k\,L_k} + e^{-px}\sum_{k=0}^{\infty}{c_k\,L_k^{'}} \\[.8em]
& L^{''} = p^2 e^{-px}\sum_{k=0}^{\infty}{c_k\,L_k} - 2 p e^{-px}\sum_{k=0}^{\infty}{c_k\,L_k^{'}} + e^{-px}\sum_{k=0}^{\infty}{c_k\,L_k^{''}}  
\end{split} 
\end{equation}\\*
Now plug in the L equation:
\begin{equation}
\begin{split}
& x(x +2)p^2 e^{-px}\sum_{k=0}^{\infty}{c_k\,L_k} - 2x(x +2)p\,e^{-px}\sum_{k=0}^{\infty}{c_k\,L_k^{'}} + x(x +2)e^{-px}\sum_{k=0}^{\infty}{c_k\,L_k^{''}} - \\[.8em]
& -(x+1)p\,e^{-px}\sum_{k=0}^{\infty}{c_k\,L_k} + (x+1)e^{-px}\sum_{k=0}^{\infty}{c_k\,L_k^{'}} + \\[.8em]
& + \left[  -p^2x^2 + (-2p^2 + 2R)x - p^2 + 2R \right] e^{-px}\sum_{k=0}^{\infty}{c_k\,L_k} = -A  e^{-px}\sum_{k=0}^{\infty}{c_k\,L_k}
\end{split}
\end{equation}\\*
Cancel exponential and $ p^2 $ terms:
\begin{equation}
\begin{split}
& - 2(x +2)p\,\sum_{k=0}^{\infty}{c_k\,xL_k^{'}}  + (x +2)\sum_{k=0}^{\infty}{c_k\,xL_k^{''}} - p\,\sum_{k=0}^{\infty}{c_k\,xL_k}  - p\,\sum_{k=0}^{\infty}{c_k\,L_k} + \\[.8em]
& + \sum_{k=0}^{\infty}{c_k\,xL_k^{'}} +  \sum_{k=0}^{\infty}{c_k\,L_k^{'}} + 2R \sum_{k=0}^{\infty}{c_k\,xL_k} + \left( - p^2 + 2R \right)\sum_{k=0}^{\infty}{c_k\,L_k} = -A  \sum_{k=0}^{\infty}{c_k\,L_k}
\end{split}
\end{equation}\\*
Expand the derivatives
\begin{equation}
\begin{split}
& - 2(x +2)p\,\sum_{k=0}^{\infty}{c_k\,k\,L_k} + 2(x +2)p\,\sum_{k=0}^{\infty}{c_{k-1}\,k\,L_{k-1}} -(x + 2)\sum_{k=0}^{\infty}{c_k\,L_k^{'}} - (x+2)\sum_{k=0}^{\infty}{c_{k-1}\,k\,L_{k-1}} - \\[.8em]
& - p\,\sum_{k=0}^{\infty}{c_k\,xL_k} - p\,\sum_{k=0}^{\infty}{c_k\,L_k} + \sum_{k=0}^{\infty}{c_k\,k\,L_k} - \sum_{k=0}^{\infty}{c_{k-1}\,k\,L_{k-1}} + \sum_{k=0}^{\infty}{c_k\,L_k^{'}} + 2R \sum_{k=0}^{\infty}{c_k\,xL_k} + \\[.8em]
&  \left( - p^2 + 2R  \right)\sum_{k=0}^{\infty}{c_k\,L_k} = -A  \sum_{k=0}^{\infty}{c_k\,L_k}
\end{split}
\end{equation}\\*
Expand again:
\begin{equation}
\begin{split}
& - 2p\,\sum_{k=0}^{\infty}{c_k\,k\,xL_k} - 4p\,\sum_{k=0}^{\infty}{c_k\,k\,L_k} + 2p\,\sum_{k=0}^{\infty}{c_{k-1}\,k,x\,L_{k-1}} + 4p\,\sum_{k=0}^{\infty}{c_{k-1}\,k\,L_{k-1}} - \sum_{k=0}^{\infty}{c_k\,k\,L_k} + \sum_{k=0}^{\infty}{c_{k-1}\,k\,L_{k-1}} -  \\[.8em]
& -2\sum_{k=0}^{\infty}{c_k\,L_k^{'}} - \sum_{k=0}^{\infty}{c_{k-1}\,k\,x\,L_{k-1}}  - 2\sum_{k=0}^{\infty}{c_{k-1}\,k\,L_{k-1}} - p\,\sum_{k=0}^{\infty}{c_k\,xL_k} - p\,\sum_{k=0}^{\infty}{c_k\,L_k} + \sum_{k=0}^{\infty}{c_k\,k\,L_k} - \sum_{k=0}^{\infty}{c_{k-1}\,k\,L_{k-1}}  + \\[.8em]
& + \sum_{k=0}^{\infty}{c_k\,L_k^{'}} + 2R \sum_{k=0}^{\infty}{c_k\,xL_k} +  \left( - p^2 + 2R  \right)\sum_{k=0}^{\infty}{c_k\,L_k} =  -A  \sum_{k=0}^{\infty}{c_k\,L_k}
\end{split}
\end{equation}\\*
Group by $ x $, $ n $, etc..:
\begin{equation}
\begin{split}
& - 2p\,\sum_{k=0}^{\infty}{c_k\,k\,xL_k} + (2p-1)\,\sum_{k=0}^{\infty}{c_{k-1}\,k\,x\,L_{k-1}}  - 4p\,\sum_{k=0}^{\infty}{c_k\,k\,L_k}+ (4p-2)\,\sum_{k=0}^{\infty}{c_{k-1}\,k\,L_{k-1}} - \\[.8em]
& - \sum_{k=0}^{\infty}{c_k\,L_k^{'}} + (2R-p)\,\sum_{k=0}^{\infty}{c_k\,xL_k}  + \left( - p^2 -p + 2R  \right)\sum_{k=0}^{\infty}{c_k\,L_k} = -A  \sum_{k=0}^{\infty}{c_k\,L_k}
\end{split}
\end{equation}\\*
Now use these formulas (from here \verb+http://mathworld.wolfram.com/AssociatedLaguerrePolynomial.html+, lines 14 and 16)
\begin{equation}
\begin{split}
\frac{d}{d\,x}L_k(x) = -\sum_{i = 0}^{k-1}{L_i(x)}
\end{split}
\end{equation}\\* 
we also need this \\*
 (from: \verb+http://www.maths.uq.edu.au/MASCOS/Orthogonal09/Warnaar.pdf+):
\begin{equation}
\begin{split}
& x\,L_k = (2k+1)L_k - (k+1)L_{k+1} - k\,L_{k-1} \\[.8em]
& x\,L_{k+1} = (2k+3)L_{k+1} - (k+2)L_{k+2} - (k+1)\,L_{k}\\[.8em]
& x\,L_{k-1} = (2k-1)L_{k-1} - k\,L_k - (k-1)\,L_{k-2} \\[.8em]
& x\,L_{k-2} = (2k-3)L_{k-2} - (k-1)\,L_{k-1} - (k-2)\,L_{n-3} \\[.8em]
\end{split} 
\end{equation}\\*
And plug in again 
\begin{equation}
\begin{split}
&  \sum_{k=0}^{\infty}{c_k  \sum_{i = 0}^{k-1}{L_i(x)} } - 2p \sum_{k=0}^{\infty}{ k(2k+1) c_kL_k} + 2p  \sum_{k=0}^{\infty}{ k(k+1) c_{k+1}L_{k+1}} +2p  \sum_{k=0}^{\infty}{ k^2  c_{k-1}L_{k-1}} + \\[.8em]
& + (2p-1) \sum_{k=0}^{\infty}{c_{k-1}\,k (2n-1)L_{k-1}} - (2p-1) \sum_{k=0}^{\infty}{c_{k}\,k^2 L_{k}} - (2p-1) \sum_{k=0}^{\infty}{c_{k-2}\,k(k-1) L_{k-2}} - \\[.8em]
& - 4p\,\sum_{k=0}^{\infty}{c_k\,k\,L_k} + (4p-2)\,\sum_{k=0}^{\infty}{c_{k-1}\,k\,L_{k-1}} + (2R-p)\,\sum_{k=0}^{\infty}{c_k\,(2k+1)L_k}  - (2R-p)\,\sum_{k=0}^{\infty}{c_{k+1}\,(k+1)L_{k+1}}  + \\[.8em]
& - (2R-p)\,\sum_{k=0}^{\infty}{c_{k-1}\,k\,L_{k-1}}  + \left( - p^2 -p + 2R  \right)\sum_{k=0}^{\infty}{c_{k}\,L_{k}} = -A  \sum_{k=0}^{\infty}{c_{k}\,L_{k}}
\end{split}
\end{equation}\\[1em]
Now group by $ n $ and for clarity remove the sum and  $ c_k $ terms:

\begin{equation}
\begin{split}
&  \sum_{k=0}^{\infty}c_k \left\{ \sum_{i = 0}^{k-1}{L_i(x)}  +  \right. \\[.8em] 
& + \left[ -2pk(2k+1) -(2p-1)k^2 -4pk +(2R-p)(2k+1) - p^2 -p + 2R \right]L_k + \\[.8em]
& + \left[2pk(k+1) - (2R-p)(k+1) \right]L_{k+1} + \\[.8em]
& + \left[2pk^2 + (2p-1)k(2k-1) + (4p-2)k - (2R-p)k \right]L_{k-1} - \\[.8em]
& \left. - \left[ (2p-1)k(k-1)  \right]L_{k-2}  \right\}
\end{split}
\end{equation}\\[1em]

Now multiply by $ L_m(x) $ and use the orthogonality of Laguerre's polynomials. The result is an 'almost' lower triangual matrix, called Hessenberg matrix.


//TODO:Remove
Once the value of $ k $ has been estimated, the vibrational energy levels are:
\begin{equation}
E_n = \hbar\left(n + \frac{1}{2}\right)\sqrt{\frac{k}{m}}
\end{equation}




