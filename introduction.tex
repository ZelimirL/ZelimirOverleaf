\fontfamily{phv}\selectfont
\setcounter{chapter}{0}
\chapter*{\textbf{Introduction}}

\label{introduction}
 
\subsection*{Motivation}

Since the quantum mechanics became current and established theory, there has been a push towards applying the quantum mechanics to the more and more complex systems. In addition, there  has been a push to apply quantum mechanical principles to processes where physics intersects with other disciplines.

Two of those disciplines are an Information Theory and Material Science.

Within Information Theory framework, the new computing paradigm is emerging, using quantum properties for the quantum information processing \cite{QIP} or more colloquially, quantum computing. The computing operations are achieved via the manipulation of the internal states of ions and atoms. Within the new materials framework, there is a great stride in creating new 2 dimensional materials (abbreviated as 2D in the text), with a lot of promising applications.\cite{Nature2D}
In order to facilitate that emergent technology it is imperative that we understand, at a fundamental level, the processes that govern the behavior of matter at the atomic level. In this thesis we will explore the interactions of ions with neutral matter. Those topics include radiative quenching, non-radiative and radiative charge transfer reactions as well as the manipulation and control of atomic matter by external fields. (TODO: Make this sentence reflect the actual work)

The quantitative calculations have greatly been helped with the advancement in computing technology, since it is now possible to solve the differential equation relatively quickly and using computing equipment which is widely accessible.  However applying only approximate methods and using only numerical methods to solve differential equations may not be sufficient in number of cases. While the numerical methods will always provide a result, they do not provide much of an insight into the behavior of a quantum system. Also, unless one is very careful, the various numerical errors tend to accumulate, and it that case the result is neither insightful nor correct. So the analytic solution is almost always desirable. Although for the moderately complex system the closed form analytic solution is usually impossible to obtain, it is still desirable to have the exact analytic solution and fall back to the approximations and numerical methods only as a last step.

The statement above is generally true and applicable to every project and solution. So in this problem we tried to find the analytical solution and fall back on the numerical computation only as a last step. So while the Quantum Mechanics can be applied to any physical system, relative to this thesis we see two major applications: new 2 dimensional materials and quantum computing.

\subsection*{New Materials}

Applying quantum mechanics to more complex systems brought us electronics components, lasers, high speed telecommunications, new diagnostic technologies in medicine, new materials, and in general, new applications are being researched.

After it discovery, at the beginning of this century, the quantum mechanics have been applied to more and more complex system. One of its application has been understanding the properties of materials.
From the understanding of the properties of the materials, came the creation of the new materials, using the quantum mechanical principles. And 2D materials were investigated fairly early in the 20th century.

\subsubsection{Graphene and Assumed Impossibility of 2D Materials}

Initially, it was argued by Landau and Pierls \cite{LandauG}\cite{Pierls} that 2D materials cannot exist. The argument was that the displacement of atoms due to the thermal fluctuations would be an order or the atomic distance, at any finite temperature \cite{LandauG, Pierls}. Thus 2D material would melt at any temperature higher that $ 0 K $. 

There is even stronger statement against the 2D materials, in the form of Mermin-Wagner-Hohenberg (MWH) theorem \cite{Hohenberg}\cite{Mermin2}. In these papers the authors show that 2D systems, with short-range interactions, are unstable, and cannot exist. 
Starting with the thermodynamics argument, for the system to be stable, its free energy must be bounded from below and convex (upwards). This way the system can minimize the energy for the ground state. Higher order fluctuations are now finite, and the system is stable.
The MWH argument is that, in 2D systems, the fluctuations around the ordered state (lattice in 2D case) decorrelate over the large distances, thus destroying the large scale order. 

Therefore, for a while it seemed that the possible existence and applications of the 2D materials were quite limited. The only 2D materials considered were a thin molecular films, which would form on the surface of solids. From Dash \cite{2DMatter1}, the 2D materials exist as a thin film either on surface on the boundaries of the 3D materials.

The other kind research in 2 dimensional systems was motivated by the need to simplify the theoretical investigations. Often, the simpler 1 or 2 dimensional conceptual models can be used to analyze the physical phenomena. From that simpler models, a 3 dimensional model is than treated as the same, just more mathematically complex system.  following those arguments, the application of  quantum mechanics in 2 dimension was initially done as an analysis of the surface phenomena. \cite{2dfilm}. 

However, despite those arguments, it has been speculated that 2D materials are possible \cite{2DMatter1}. The situation has changed dramatically in 2004 when it turns out that 2D materials can indeed exist \cite{Graphene0}. The first 2D material, Graphene was created \cite{GrapheneN} at the University of Manchester, Great Britain.

The 2D materials have qualitatively different topological properties \cite{2DMatterCurvature}. Moreover, these properties do not depend on the type of microscopic interactions between particles. While this subject is beyond the scope of this thesis, it does show that the field of 2D materials has excellent potential for both fundamental research and applications. For the material to exhibit 3D properties, it has been shown \cite{GraphLayers} that when material has as low as 10 layers, its properties are approaching the 3D limit.

With regard to the above mentioned impossibility theorem \cite{Mermin2}, the Graphene does not contradict it, but it does circumvent it \cite{GrapheneRiples}. In Graphene thermal oscillation produce ripples, or bending of the Graphene sheet. That way Graphene behaves like an elastic membrane. That allows phonon to propagate in 2 dimensions, and couple the in plane stretching to the transverse fluctuations out of the plane. This mediates a long range interaction, thereby circumventing Mermin-Wagner 

Interestingly, following \cite{Graphene0} it seems that 2D materials have another set of interesting properties, where particles in 2D behave relativistically. Usually, in condensed mater system, the ordinary Schrodinger equation is sufficient to describe all the electronic properties of the material. In Graphene, the electrons themselves still behave non-relativistically. But there interaction with the periodic potential of the Graphene lattice gives rise to new quasi-particles, known as the massless Dirac's fermions. These quasi-particles, even at low energies, can only be described by the Dirac's equation in 2+1 dimensions. Therefore it might be interesting to redo the same calculation as we did, but solving the Dirac's equation for the system of quasi-particles similar to an Hydrogen ion.  There are many other interesting properties of Graphene \cite{Graphene0}, such as its QED (QuantumElectroDynamic) like electronic spectrum, electron tunneling, and so on. The other interesting consequence of the QED like electron spectrum is the possibility of experimentally studying QED in curved space, by controllable bending of a Graphene sheet. This can offer a possibility to address a certain class of cosmological problems. 

After the discovery of Graphene, there has been an extensive study regarding the new 2D materials. and the research has really picked up. Now there are number of new 2D materials being discovered \cite{Many2DMaterials}.  The main application is in the direction of nanotubes and 2D semiconductors. While nanotubes are made of Graphene, they are not equivalent to each other,  and depending on the application Graphene (as a sheet) is sometimes superior, sometimes inferior and sometimes completely different. The other very exciting  application is the the appearance of high temperature superconductivity in 2D materials\cite{2DSuper}.

\subsubsection*{2D Materials}

Interestingly, following \cite{Graphene0} it seems that 2D materials have another set of interesting properties, in the regions where particles in 2D behave relativistically. Usually, in condensed mater system, the ordinary Schrodinger equation is sufficient to describe all the electronic properties of the material. In Graphene, the electrons themselves behave somewhat relativistically, moving with velocities of $ \approx 10^6 m/s $. While this is still 300 times slower than the speed of light, it is still much faster that the electron velocities in a classical conductor, which are the order of $ 10^(-4) m/s $.

However the interaction of conducting electrons in Graphene with the periodic potential of the Graphene lattice gives rise to new quasi-particles, which behave as massless Dirac's fermions. These quasi-particles, even at low energies, can only be described by the Dirac's equation in 2+1 dimensions. Therefore it might be interesting to redo the same calculation as we did, but solving the Dirac's equation for the system of quasi-particles similar to an Hydrogen ion.  There are many other interesting properties of Graphene \cite{Graphene0}, such as its QED like electronic spectrum, electron tunneling, and so on. The other interesting consequence of the QED like electron spectrum is the possibility of experimentally studying QED in curved space, by controllable bending of a Graphene sheet. This can offer a possibility to address a certain class of cosmological problems. 

\subsubsection*{2D Electronics}

One application of the 2D materials is their use in electronic devices \cite{2DEJour1},\cite{2DEJour2}. This is driven by the demand for the higher performance and lower consumption in electronic devices.
Regarding the applications in electronics, the hope is that the Graphene can be used to build active electronic components, such as diodes, transistors (FET) and other electronic components. Alternatively due to its conductivity, Graphene can be used as a conductor, in both electronic components but also in the batteries. Graphene itself is a zero gap semiconductor. There is another unique opportunity in 2D semiconductors. In 3D semiconductor electronic states are buried deep inside the material. In 2D semiconductors, the electronic states are on the surface. The more interesting application of 2D electronics are the LED elements in 2D \cite{2DLED}. The research in electronics is closely related to the research of new 2D materials. For example Graphene itself is not compatible with Silicon. But there is a new material, Black Phosphorus \cite{2DPhos}, which is both compatible with Silicon and it holds promise for the future electronics \cite{2DPhos3}. That's due to two key properties. One is that black phosphorus has a higher mobility than silicon, which is the speed at which it can carry an electrical charge. The other is that it has a bandgap, which Graphene does not. So in essence, in the presence of an electric field black phosphorus acts as a semiconductor \cite{2DPhos2}.

\subsection*{Quantum Computing}

The current computer architecture is based on so called 'Von Neumann' architecture. The algorithms based on that architecture have some inherent limits, and as it currently stands, the architecture cannot overcome those limits.

There are two main  classes of computer algorithms, P and NP. There are subclasses within those classes, but those two classes capture the essential behaviour of the belonging algorithms. The algorithms belonging to class P, are those which are solvable in polynomial time, with regards to the size of the input. Those algorithms belonging to class NP are those which require exponential time to complete, again with regards to the input size.\cite{PvsNP}

Currently the greatest unsolved problem in Computer Science is the P vs NP problem \cite{PvsNP}. While hypothesis that $ P \neq NP $ is generally accepted, no proof has been found. Applying quantum mechanical principles in engineering disciplines, such as computer science \cite{FQC} promises to bring us the much qualitatively more powerful quantum computer. The theoretical basis of quantum computation is described by the Quantum Turing Machine \cite{QuantumTuring}.

Currently it is not clear whether the quantum computer will be able to solve NP problems \cite{QCvsNP}. Now all problems in the NP class (more precisely NP complete subclass) can be transformed into each other in polynomial time to each other \cite{NPComplete},  therefore finding a polynomial time solution for one of them would amount to having a polynomial time solution to the all problems in NP class. So the goal is rather worthy.

Currently, there are not final word on the subject yet. While most authors lean to the opinion that the quantum computer may not be fundamentally more powerful than the classical one, there are schemes \cite{NLQC} which would allow quantum computer to solve NP problems. As an related example the Schor's algorithm \cite{Schor} can provably solve integer factorization in a polynomial time, as opposed to the best classical algorithm (quadratic sieve), which works in sub-exponential time \cite{Pomerance}. Unfortunately, the integer factorization problem does not belong to the class of NP complete problems. 

Even if the NP problems remain intractable, just the possibility that, according to Feynman \cite{FQC}, quantum computer can be used to simulate the quantum processes,will in itself be a step forward in modeling the physical properties and processes. In addition quantum computers may offer some other benefits in terms of space, energy consumption.

\subsection*{Topological Quantum Computing}

What relates the new 2D materials and the quantum computing architecture, is the concept of topological quantum computing \cite{Tqc1}.  While, in principle, there are no theoretical objections to building a quantum computer, there are serious issues encountered in practice. \cite{QCProblems}. Currently these problems are not solved and the quantum computers currently in existence are still unable to overcome them \cite{QCProblems}.
It seems that the topological approach promises an actually physically realizable quantum computer \cite{Tqc2,Tqc3}. 


\subsection*{Anyons}

Anyons are the quasi-particles which occur only in 2D systems and whose quantum statistics in neither Fermionic nor Bosonic \cite{Anyons1}.  This has been proved  \cite{Anyons2}, and the arbitrary statistics is valid only in 2 dimensions \cite{Walsh}. After exchanging two identical particles, the wave function gains an arbitrary phase factor $ \Psi = e^{i\theta}\Psi $, where $ \theta = 2\pi\nu^{s} $. This is in contrast the 3 dimensional world, where exchange of two particles gains either phase factor $ \pi $ or $ 0 $, for Fermions and Bosons respectively.

In general, there are two kinds of Anyons, Abelian and non-Abelian. The Abelian Anyons have been reliably detected in the Fractional Quantum Hall Effect. \cite{FQHE}.  While the computation with Abelian Anyons is theoretically possible \cite{AbelianAnyons}, considerably more attention has been paid to the non-Abelian Anyons.
There are several possibilities for realizing non-abelian statistics experimentally: 1. a two-dimensional electron gas in a large magnetic field (the fractional quantum Hall effect); 2. rapidly rotating Bose-Einstein condensates \cite{RrBeC} 3. frustrated magnets \cite{FrMag}.

Non-abelian Anyons have not been detected yet, but most of the research have been focused on them as  they could offer some interesting applications, most notably  fault tolerant Topological Quantum Computing. \cite{AnyonsTqc}. But since the Anyons are only observed in 2D topologies, one must start analyzing the 2D materials.


\subsection*{Excitons and Trions}

The other interesting and relatively simple system is the Exciton \cite{Excitons2D1}, which represents a bound state of a whole and a electron in a solid state material, and allows the transport of energy without transport of an electric charge. There are two types of Excitons: 

1. Frenkel exciton which is found in organic molecular crystals \cite{Excitons3} with binding energy of $ 1eV $ and radius of $ ~ 10\text{\AA} $ and 
2. Wannier exciton, found in in semiconductors \cite{Excitons2}  with binding energy of $ ~1meV $ and radius of $ ~100\text{\AA} $

Exciton have been observed in 2D materials \cite{Excitons2D2} and also as 2D Excitons in 3D materials. The promise here is that these kind of quasi-particles can help create a new kind of electronic components, typically opto-electronics.

Since the exciton system resembles either the Hydrogen atom or the Hydrogen molecule, and it seems conceivable that the dynamics of  $ H2^{+} $ molecule is qualitatively similar and could be applied to the excitons systems as well.

The similar system is a Trion, which consists of either 2 electrons and a hole, or 2 holes and an electron, therefore resembling the hydrogen molecule.

\subsection*{Future Applications and Speculations}

While the applications of the 2D materials above are certainly interesting, creating an exact analytical  model of such material is extremely complex. Even with today's computing machinery, the wave equations for many bodies quantum system are unsolvable, even numerically.  Equations for many bodies systems may formally be created by extending equations for a one or two bodies systems. But as soon as one tries that, one realizes that the complexity of such equations increases enormously, and than assumptions  made for a small systems are not valid any more. In this author's opinion, moving from a simple system to an order of magnitude more complex, requires a completely new mathematical and physical framework. Something analogous between the classical mechanics and thermodynamics.

\subsection*{Outline}

In this thesis we will focus on investigating and solving the Radiative Charge Transfer TODO: (Reflect actual work) for the simplest molecular system, namely Hydrogen Molecular Ion $ {H_2^{+}}  $ in 2 dimensions (2D in the further text), and solving such system in 2D exactly.  In chapter 1 we will show that applying our method to the already solved and well known $ {H_2^{+}} $ in 3D yields the same results. Then in chapter 2 we will follow  the already well established procedure \cite{Bates1} to analytically solve the $ {H_2^{+}} $. After providing an analytical solution, we will fall back to the numerical methods, in order to calculate the energy levels, and obtain the potential curves. In chapter 3 we use the results obtained to calculate the dynamics of such system, namely the Radiate Association process. In chapter 4 we again use the same results to calculate the Radiate Charge Transfer. TODO: (See above TODO)

In the following chapter, we re-calculate the electron energies for Hydrogen Molecular Ion in 3D, by using different and arguably simpler approach.

